\paragraph{Résumé du chapitre}

Nous avons vu dans le chapitre précédent que les classifieurs analogiques
proposaient des résultats prometteurs dans des domaines booléens,
comparativement aux classifieurs $k$-NN. Nous nous intéressons, dans ce
chapitre et dans le suivant, à l'usage de l'apprentissage par analogie à des
tâches plus concrètes, telles que celle de la recommandation.

Les systèmes de recommandation sont des outils de filtrage automatique qui
permettent de fournir des suggestions d'items aux utilisateurs d'un système.
Ils permettent de répondre à des questions aussi variées que ``dans quel
restaurant manger?'', ``quel film visionner?'', etc. Pour cela, les systèmes de
recommandation fournissent des listes de suggestions aux utilisateurs.
Le but de ce chapitre est de donner les prérequis nécessaires au sujet de
systèmes de recommandation, en vue du chapitre suivant où nous décrirons nos
contributions à la recommandation analogique.

Nous avons vu qu'il existe trois familles principales de systèmes de
recommandation : les méthodes \textit{content-based}, les méthodes par filtrage
collaboratif, et les méthodes dites \textit{knowledge-based}. Le problème que
nous nous proposons d'étudier ici est celui de la prédiction de notes : étant
donné un ensemble de notes passées caractérisant les interactions d'un groupe
d'utilisateurs avec un groupe d'items, le but d'un algorithme de prédiction est
d'estimer les notes manquantes. Pour ce genre de problèmes, les systèmes
collaboratifs sont généralement nettement plus performants que les méthodes
\textit{content-based}. Nos contributions seront donc de nature collaborative.

Nous avons décrit en détail deux techniques populaires de filtrage
collaboratif. La première est la technique par voisinage, qui est une
généralisation directe des méthodes $k$-NN. Pour prédire la note d'un
utilisateur pour un item, le \textit{voisinage} de l'utilisateur est estimé,
puis on procède à une agrégation des notes des voisins pour l'item cible. Le
voisinage est généralement estimé à l'aide d'une mesure de similarité qui
caractérise à quel point deux utilisateurs ont tendance à noter les items avec
les mêmes notes. Ce genre de méthode a tendance à  modéliser des interactions
locales sous-jacentes aux données, contrairement aux méthodes par factorisation
de matrice qui modélisent des effets globaux. Les techniques par factorisation
de matrice modélisent les utilisateurs et les items comme des vecteurs de
\textit{facteurs latents}, lesquels sont estimés via un problème
d'optimisation. La note d'un utilisateur pour un item est alors donnée par le
produit scalaire entre leurs deux vecteurs respectifs. Ces deux familles de
méthodes (voisinage et factorisation de matrice) nous serviront à comparer les
performances de nos algorithmes dans le prochain chapitre.
