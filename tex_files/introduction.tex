\chapter*{Introduction}
\addcontentsline{toc}{chapter}{Introduction} % To add to TOC
\initial{A}nalogical reasoning is widely recognized as a powerful ability of
human intelligence.  It can lead to conclusions for new situations by
establishing links between apparently unrelated domains. One well-known example
is the Rutherford-Bohr model of atom where electrons circle around the kernel,
which is analogically linked to the model of planets running around the sun.
Unsurprisingly, this kind of reasoning has generated a lot of attention from
the artificial intelligence community.

In this document we will focus on a particular model of analogical reasoning,
based on \textbf{analogical proportions}. An analogical proportion is a
statement of the form \textit{a is to b as c is to d}, and expresses the fact
$a$ differs from $b$ in the same manner as $c$ differs from $d$. For example,
one could say that \textit{France is to Paris as Germany is to Berlin}, or that
\textit{electrons are to the kernel as planets are to the sun}. In some cases,
when the element $d$ of the proportion is not known\footnote{Or any other
element, actually.}, it can be \textbf{inferred} from the three other elements
$a, b, c$. It seems indeed natural that even a child with basic geographic
knowledge could answer the question \textit{France is to Paris as Germany is to
what?} with the correct answer: \textit{Berlin}. And even in the case where our
subject does not know that the name of the correct answer is \textit{Berlin},
they could still \textbf{imagine} a city  which is the capital of Germany, and
suppose that this hypothetical city corresponds to the correct answer. We are
here witnessing two key cognitive processes that can triggered by the use
of analogical proportions: inference, and creativity.

As a tool that allows inference, analogical reasoning has been used for machine
learning purposes. Indeed, analogical proportion-based learning has been
addressed in recent works: we can cite \cite{StrYvoCNLL05} for an application
in linguistics (verb conjugation task), and \cite{BayMicDelIJCAI07} for
classification problems in a Boolean setting. Our investigations completely
fall within the scope of these two previous works, and the goal of this thesis
is twofold. The first, topic is to \textbf{exhibit some theoretical properties
of analogical classifiers}, and the second one is to \textbf{apply analogical
reasoning to real-world problems}.

Our theoretical investigations are motivated from the fact that so far,
analogical classifiers were only known from their algorithmic descriptions.
In fact, each implemented classifier provides a clean description
of {\it how to compute}, but we definitely miss a clean description of {\it
what do we actually compute}. The previous experimental investigations were not
directed to provide an analytical view, and as a result analogical classifiers
were yet quite poorly understood, and we had no clear knowledge of their
strengths and weaknesses.

Also, the application of analogical reasoning to concrete tasks is motivated
from the fact that the previous empirical investigations were only led in
standard artificial settings. To our knowledge, the analogical learners were yet
to be used in real-world problems to assess their suitability in concrete
applications, where the requirements for success are often significantly
different. We chose here to apply analogical learning to the field of
recommender systems.

\paragraph{Road map and contributions\\}

Chronologically, we first focused on the applicative side. Our first
contributions were indeed devoted to applying analogical reasoning to the task
of recommendation. Then, later, we led our theoretical investigations about
analogical learners.

In this document, we will choose however to present our work neither
chronologically, nor thematically. We will instead first present some of our
theoretical results (Chapter \ref{CHAP:functional_definition}), then we will
fully describe our contributions to analogical recommendation (Chapter
\ref{CHAP:analogical_recommendation}), and finally we will go back to other
theoretical considerations in Chapter \ref{CHAP:analogy_preserving_functions}.
The remaining chapters
(\ref{CHAP:computational_models_of_analogical_reasoning},
\ref{CHAP:formal_analogical_proportions}, and
\ref{CHAP:background_reco_systems}) are intended to provide the necessary
background on the different topics. This choice, that may seem a bit puzzling
at first sight, is in fact motivated by pedagogical reasons. It will indeed
become clear that the topic of analogical recommendation (Chapter
\ref{CHAP:background_reco_systems}) is a lot easier to
introduce and to understand once we have enough background on analogical
classification (Chapter \ref{CHAP:functional_definition}), as our methods for
analogical recommendation are strongly inspired by previous works on analogical
classification. Moreover, the motivation for Chapter
\ref{CHAP:analogy_preserving_functions} can be directly derived from the
results of Chapter \ref{CHAP:functional_definition}, but it will be also
interesting to account for \ref{CHAP:analogy_preserving_functions} in the light
of the result of Chapter \ref{CHAP:functional_definition}.
This document is structured as follows.\\

The first chapter will provide the necessary background on existing
models of analogical reasoning, with a strong emphasis on models that allow to
perform computational inference. These models are, for the most part, motivated
by cognitive and psychological aspects, which is in contrast to our main
computational tool: formal analogical proportions.\\

Formal analogical proportions are the subject of the second chapter. We will
provide the definition of analogical proportions in various algebraic settings,
focusing on the two proportions that we will mostly use: the arithmetic
proportion (dealing with real numbers), and the Boolean proportion. In addition
to providing the formal definitions, we will try to give a geometrical insight
on these two proportions, which to the best of our knowledge had never been
done before. We will also go through a toy classification problem that will
allow us to describe the process of \textbf{analogical equation
solving}\footnote{Very roughly, analogical equation solving is the process of
finding the unknown $x$ in the proportion \textit{$a$ is to $b$ as $c$ is to
$x$}.}, and the \textbf{analogical inference principle} which is the principle
underlying all of our investigations. In a classification context, the
analogical inference principle states that if $a$ is to $b$ as $c$ is to $d$,
then we should also have that $f(a)$ is to $f(b)$ as $f(c)$ is to $f(d)$, where
$f(x)$ is the function that determines the class of $x$. This is obviously an
unsound inference principle, in that the conclusion does not follow from the
premise, but it can still be used for classification tasks which is the subject
of the third chapter.\\

The study of analogical classification is indeed the object of Chapter
\ref{CHAP:functional_definition} which corresponds to a detailed version of our
ECAI paper: \cite{HugPraRicSerECAI16}. In this chapter, we will start to
address one of the two main objectives of this thesis: identifying theoretical
properties of analogical classifiers. Our first contribution will be to provide
a functional definition of analogical classifiers, that will unify the two
previous approaches mentioned earlier. From this definition, we will be able to
derive results related to the VC-dimension of these classifiers, as well as
their error rate. Our functional definition will also reveal the close links
that connect the analogical learners to the $k$-nearest neighbors ($k$-NN)
techniques. We will show indeed that the analogical classification process can
be viewed as a two-step procedure that first consists in extending the training
set (by \textbf{generating} new examples), and then applying a $k$-NN
classifier. Quite remarkably, the two key cognitive processes related to
analogical proportions that we have mentioned, namely inference and creativity,
are here blended together. From these results, a natural question arises: how
can we ensure a training set extension that is completely error-free? This
question will be addressed later in Chapter
\ref{CHAP:analogy_preserving_functions}, and the next two chapters
(\ref{CHAP:background_reco_systems} and \ref{CHAP:analogical_recommendation})
will be devoted to our second main objective: applying analogical reasoning to
the task of recommendation.\\

Chapter \ref{CHAP:background_reco_systems} will be dedicated to the necessary
background on recommender systems. We will briefly review the three main
families of recommender systems, namely content-based techniques, collaborative
filtering, and knowledge-based systems. We will also formally define the
problem that we plan to address, which is that of rating prediction: given a
database of  user-item interactions taking the form of ratings, the goal is to
predict all the ratings for the pairs (user, item) that are not in the
database. The different measures allowing to assess the quality of the
prediction will be presented. Finally, as our contributions will be of a
collaborative nature, we will thoroughly detail two popular methods for
collaborative filtering: the neighborhood-based techniques, and the
matrix-factorization-based methods. These two families of algorithms will serve
as benchmarks to compare the performance of our own algorithms.\\

In Chapter \ref{CHAP:analogical_recommendation}, we will present our
contributions to the topic of recommender systems. We will first describe our
preliminarily investigations, which are a direct adaptation of the analogical
classifier described in the previous chapter. These first investigations were
the object of a paper published at ISMIS \cite{HugPraRicISMIS15}. It will
become clear that while offering similar accuracy to the traditional
approaches, analogical recommender suffer from their cubic complexity.
Acknowledging this fact, we developed another view of analogical
recommendation, taking into account the fact that some users may have different
interpretation of the rating scale. These investigations where described in a
Fuzz-IEEE paper \cite{HugPraRicSerFuzzIEEE16}. Finally, we will address the
problem of mining analogical proportions between users (or items) in a rating
database \cite{HugPraRicSerLFA16}. Our result will help us to retrospectively
interpret the modest performances of our analogical recommenders: as bland as
it may seem, it turns out that there probably just was not enough decent
analogies to be found in the databases that we used.\\

But this down-to-earth observation will not spare us the very questioning of
the analogical inference principle that had been underlying all of our
investigations so far. In the last chapter, we will go back to our theoretical
considerations, and provide a criteria that allows to apply this analogical
inference in a sound way. We have briefly described that the analogical
inference principle states that if four elements $a, b, c, d$ are in proportion,
then their classes $f(a), f(b), f(c), f(d)$ should also be in proportion. We
will provide a complete characterization of the functions $f$ such that when
four elements are in proportion, then their image by $f$ are also in
proportion. It will be clear that these functions are also the ones that allow
to extend a training set by analogy in a perfectly safe way. We will call these
functions the \textbf{analogy preserving} functions, and they will turn out to
be the well-known affine functions, both in Boolean and real settings.
