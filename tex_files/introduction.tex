\chapter*{Introduction}
\addcontentsline{toc}{chapter}{Introduction} % To add to TOC
\initial{A}nalogical reasoning is widely recognized as a powerful ability of human
intelligence.  It can lead to conclusions for new situations by establishing
links between apparently unrelated domains. One well known example is the
Bohr's model of atom where electrons circle around the kernel, which is
analogically linked to the model of planets running around the sun. It is not
surprising that this kind of reasoning has generated a lot of attention from
the artificial intelligence community.


Nevertheless, all these theoretical investigations are not directed to provide
an analytical view of analogy-based learners. In that sense, they are not
really helpful if we want to characterize the behaviour of an analogical
classifier for instance. One of the reasons could be that, unlike the $k$-NN
rule, the analogical learning rule is not easily amenable to a functional
definition. In fact, each implemented algorithm  provides a clean description
of {\it how to compute} but we definitely miss a clean description of {\it what
do we actually compute}.
Since such a definition, even a simplified one, is paramount to investigate
theoretical properties, we suggest here a concise functional definition and we
prove that it fits with the main implementations of analogical classifiers.

\paragraph{Road map\\}


