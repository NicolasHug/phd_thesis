\chapter*{Introduction}
\addcontentsline{toc}{chapter}{Introduction} % To add to TOC
\initial{A}nalogical reasoning is widely recognized as a powerful ability of
human intelligence.  It can lead to potential conclusions for new situations by
establishing links between apparently unrelated domains. One well-known example
is the Rutherford-Bohr model of atom where electrons circle around the kernel,
which was analogically inspired by the model of planets running around the sun.
Unsurprisingly, this kind of reasoning has generated a lot of attention from
the artificial intelligence community.

In this work we will focus on a particular model of analogical reasoning,
based on \textbf{analogical proportions}. An analogical proportion is a
statement of the form \textit{a is to b as c is to d}, and expresses the fact
$a$ differs from $b$ in the same manner as $c$ differs from $d$. For example,
one could say that \textit{France is to Paris as Germany is to Berlin}, or that
\textit{electrons are to the kernel as planets are to the sun}. In some cases,
when the element $d$ of the proportion is not known\footnote{Or any other
element, actually.}, it can be \textbf{inferred} from the three other elements
$a, b, c$. It seems indeed natural that even a child with basic geographic
knowledge could answer the question \textit{France is to Paris as Germany is to
what?} with the correct answer: \textit{Berlin}. And even in the event that our
subject does not know that the name of the correct answer is \textit{Berlin},
they could still \textbf{imagine} a city  which is the capital of Germany, and
suppose that this hypothetical city corresponds to the correct answer. We
witness here two key cognitive processes that can be triggered using analogical
proportions: \textbf{inference}, and \textbf{creativity}.

As a tool that allows inductive inference, analogical reasoning has been
proposed for plausible reasoning and for machine learning purposes. Indeed,
analogical proportion-based learning has been addressed in recent works: we may
cite for instance \cite{StrYvoCNLL05} for an application in linguistics (verb
conjugation task), and \cite{BayMicDelIJCAI07} for classification problems in a
Boolean setting. Our investigations follow on from these two works, and the
goal of this thesis is twofold. The first topic is to \textbf{apply analogical
reasoning to real-world problems}, and the second one is to \textbf{exhibit
some theoretical properties of analogical classifiers}.


The application of analogical reasoning to concrete tasks is motivated by the
fact that the previous empirical investigations were only led in standard
artificial settings. To our knowledge, the analogical learners were still to be
experienced in real-world problems to assess their suitability in concrete
applications, where the requirements for success are often significantly
different. We have chosen here to apply analogical learning to the field of
recommender systems.

Also, our theoretical investigations are motivated by the fact that so far,
analogical classifiers were only known from their algorithmic descriptions.  In
fact, each implemented classifier provides a clean description of {\it how to
compute}, but we definitely miss a clean description of {\it what do we
actually compute}. The previous experimental investigations were not directed
to provide an analytical view, and as a result analogical classifiers were yet
quite poorly understood: we had no clear knowledge about their strengths and
weaknesses.

\paragraph{Road map and contributions\\}

Chronologically, we first focused on the applicative side. Our first
contributions were indeed devoted to applying analogical reasoning to the task
of recommendation. Later, we led our theoretical investigations on
analogical classification.

In this document, we will choose to present our work somehow differently.
After having recalled the necessary background on analogical proportions in
Chapters \ref{CHAP:computational_models_of_analogical_reasoning} and
\ref{CHAP:formal_analogical_proportions}, we
will present some of our
theoretical results in Chapter \ref{CHAP:functional_definition}. Then, we will
fully describe our contributions to analogical recommendation  in Chapters
\ref{CHAP:background_reco_systems} and \ref{CHAP:analogical_recommendation},
and finally we will address more theoretical considerations in Chapter
\ref{CHAP:analogy_preserving_functions}. Simply put, our contributions to the
theoretical side will be split up by the topic of analogical recommendation.

This choice is in fact motivated by pedagogical reasons. It will
indeed become clear that the topic of analogical recommendation (Chapters
\ref{CHAP:background_reco_systems} and \ref{CHAP:analogical_recommendation}) is
a lot easier to introduce and to understand once we have enough background on
analogical classification (Chapter
\ref{CHAP:functional_definition}), as our methods for analogical recommendation
are strongly inspired by previous works on analogical classification. Moreover,
whereas the motivations for Chapter \ref{CHAP:analogy_preserving_functions}
could have been directly derived from the results of Chapter
\ref{CHAP:functional_definition}, it will be more interesting to account for
Chapter \ref{CHAP:analogy_preserving_functions} in the light of the results of
Chapter \ref{CHAP:analogical_recommendation}.

We now provide the detailed structure of this document.\\

The first chapter will provide the necessary background on existing
models of analogical reasoning, with a strong emphasis on models that allow to
perform computational inference. It will appear that these models are, for the
most part, motivated by cognitive and psychological aspects, and depart from
our main computational tool: formal analogical proportions.\\

Formal analogical proportions are the subject of the second chapter. We will
provide their definitions in various algebraic settings,
focusing on the two proportions that we will mostly use: the arithmetic
proportion (dealing with real numbers), and the Boolean proportion. In
addition, we will try to give a geometrical insight on these two proportions,
which to the best of our knowledge had never been
done before. We will also go through a toy classification problem that will
allow us to describe the process of \textbf{analogical equation
solving}\footnote{Very roughly, analogical equation solving is the process of
finding the unknown $x$ in the proportion \textit{$a$ is to $b$ as $c$ is to
$x$}, e.g. \textit{France is to Paris as Germany is to What?}}, and the
\textbf{analogical inference principle} that underlies all
of our investigations. In a classification context, this
principle states that if $a$ is to $b$ as $c$ is to $d$,
then we should also have that $f(a)$ is to $f(b)$ as $f(c)$ is to $f(d)$, where
$f(x)$ is the function that determines the class of $x$. It is obviously an
unsound inference principle, in that the conclusion does not follow from the
premise. But as we will see, it can still be used for classification tasks,
which is the subject of the third chapter.\\

The study of analogical classification is indeed the object of Chapter
\ref{CHAP:functional_definition}, which corresponds to a detailed version of our
ECAI paper: \cite{HugPraRicSerECAI16}. In this chapter, we will start to
address one of the two main objectives of this thesis: identifying theoretical
properties of analogical classifiers. Our first contribution will be to provide
a functional definition of analogical classifiers, that will unify the two
pre-existing approaches mentioned earlier (\cite{StrYvoCNLL05} and
\cite{BayMicDelIJCAI07}). This new definition will bring more
insight into these classifiers, enabling us to derive results related to their
Vapnik-Chervonenkis dimension, as well as their error rate. Our functional
definition will also
reveal the close links between the analogical classifiers and the $k$-nearest
neighbors ($k$-NN) techniques. We will show indeed that the analogical
classification process can be viewed as a two-step procedure that first
consists in extending the training
set (by \textbf{generating} new examples), and then applying a $k$-NN
classifier. Quite remarkably, the two key cognitive processes related to
analogical proportions (inference and creativity),
are here blended together. From these results, a natural question arises: how
can we ensure a training set extension that is completely error-free? This
question will be addressed later in Chapter
\ref{CHAP:analogy_preserving_functions}, while the  following two chapters
(\ref{CHAP:background_reco_systems} and \ref{CHAP:analogical_recommendation})
will be devoted to our second main objective: applying analogical reasoning to
the task of recommendation.\\

Chapter \ref{CHAP:background_reco_systems} will be dedicated to the necessary
background on recommender systems. We will briefly review the three main
families of recommender systems, namely content-based techniques, collaborative
filtering, and knowledge-based systems. We will also formally define the
problem that we plan to address, which is that of rating prediction: given a
database of  user-item interactions taking the form of ratings, the goal is to
predict all the ratings for the pairs (user, item) that are not in the
database. The different measures allowing to assess the quality of the
predictions will be presented. Finally, as our contributions will be of a
collaborative nature, we will thoroughly detail two popular methods for
collaborative filtering: the neighborhood-based techniques, and the
matrix-factorization-based methods. These two families of algorithms will serve
as benchmarks to compare the performance of our own algorithms.\\

In Chapter \ref{CHAP:analogical_recommendation}, we will present our
contributions to the topic of recommender systems. We will first describe our
preliminarily investigations, which are a direct adaptation of the analogical
classifier described in the previous chapter. These first investigations were
the object of a paper published at ISMIS \cite{HugPraRicISMIS15}. It will
become clear that while offering similar accuracy to the traditional
approaches, analogical recommenders suffer from their cubic complexity.
Acknowledging this fact, we developed another view of analogical
recommendation, taking into account the fact that some users may have different
interpretation of the rating scale. The algorithms that we derived were much
more scalable than the previous ones, and these investigations were described
in a Fuzz-IEEE paper \cite{HugPraRicSerFuzzIEEE16}.  Finally, we will address
the problem of mining analogical proportions between users (or items) in a
rating database \cite{HugPraRicSerLFA16}. Our results will help us to
retrospectively interpret the modest performances of our analogical
recommenders: as bland as it may seem, it turns out that there were just not
enough decent analogies to be found in the available databases.\\

But this down-to-earth observation will lead us to the very questioning of
the analogical inference principle that had been underlying all of our
investigations so far. In the last chapter, we will go back to our theoretical
considerations, and provide a criterion that allows us to apply analogical
inference in a sound way. We have briefly described that the analogical
inference principle states that if four elements $a, b, c, d$ are in
proportion, then their classes $f(a), f(b), f(c), f(d)$ should also be in
proportion. We will provide a complete characterization of the functions $f$
such that when four elements are in proportion, then their image by $f$ are
also in proportion: these functions ensure a safe use of the analogical
inference principle. It will be clear that these functions are also the ones
that allow to extend a training set by analogy in a perfectly safe way. We will
call these functions the \textbf{analogy preserving} functions, and they will
turn out to be the well-known affine functions, both in Boolean and real
settings. These investigations led to an IJCAI paper
\cite{CouHugPraRicIJCAI17}, and provide various future research tracks.\\


We will now start with the first chapter, dedicated to the description of
previous (and current) attempts at formalizing analogical reasoning.

\newpage

\section*{Introduction in French}

Le raisonnement par analogie est largement considéré comme une caractéristique
clef de l'intelligence humaine. Il permet d'arriver à de potentielles
conclusions pour de nouvelles situations, en établissant des liens entre des
domaines en apparence très différents. Un exemple connu  est celui du modèle de
l'atome de Rutherford-Bohr, où les électrons gravitent autour du noyau, comme
les planètes du système solaire gravitent autour du soleil. Ce type de
raisonnement a fait l'objet de nombreuses études de la part des psychologues et
des philosophes, mais il intéresse aussi la communauté de l'intelligence
artificielle.

Ce travail s'intéresse à un mode particulier de raisonnement par analogie, à
savoir le raisonnement par proportions analogiques. Une proportion analogique
est une proposition de la forme \textit{a est à b comme c est à d}, et exprime
le fait que $a$ diffère de $b$ comme $c$ diffère de $d$. On pourrait par
exemple affirmer que la \textit{France est à Paris ce que l'Allemagne est à
Berlin}, ou encore que \textit{les électrons sont au noyau ce que les planètes
sont au soleil}. Dans certains cas, lorsque l'élément $d$ n'est pas connu, on
pourra l'inférer à partir des trois autres éléments $a$, $b$, et $c$. Il semble
en effet naturel qu'un enfant doté  de connaissances géographiques basiques
pourrait répondre à la question \textit{La France est Paris ce que l'Allemagne
est à quoi ?} Et même dans l'éventualité où l'enfant ne connaîtrait pas le nom
de la ville \textit{Berlin}, il pourrait tout de même s'imaginer une
hypothétique ville au nom inconnu, mais qui serait la capitale de l'Allemagne.
L'utilisation des proportions analogiques nous permet ici de toucher du doigt
deux processus cognitifs clef : l'inférence, et la créativité.

En tant qu'outil permettant une forme d'inférence inductive, le raisonnement
par analogie a déjà été envisagé pour des problèmes d'apprentissage artificiel.
On pourra citer par exemple \cite{StrYvoCNLL05} pour une application en
linguistique, ainsi que \cite{BayMicDelIJCAI07} pour des problèmes de
classification dans des domaines booléens. Nos recherches s'inscrivent
dans la lignée de ces deux précédents travaux, et le but de cette thèse est
double. Le premier sujet de recherche est d'appliquer le raisonnement
analogique à des problèmes concrets, et le second est d'étudier les classifieurs
analogiques d'un point de vue théorique, afin d'en exhiber des propriété
théoriques intéressantes.

L'application du raisonnement analogique à des problèmes concrets est motivée
par le fait que les précédentes recherches ont principalement été menées d'un
point de vue expérimental, dans des environnements assez artificiels. A notre
connaissance, aucun travail n'avait été mené pour évaluer la pertinence des
classifieurs analogiques sur des tâches concrètes, où les critères de succès
sont souvent différents. Nous avons ici choisi d'appliquer le raisonnement
analogique au problème de la recommandation, et plus particulièrement à celui
de la prédiction de notes.

D'autre part, nos recherches théoriques sont motivées par le fait que
jusqu'alors, les classifieurs analogiques n'étaient connus que via leurs
descriptions algorithmiques. On avait une connaissance claire de comment
calculer la sortie d'un classifieur analogique, mais la nature de ce qui était
effectivement calculé restait encore assez floue. On manquait alors d'outils
pour caractériser les forces et les faiblesses de tels classifieurs.

\paragraph{Annonce du plan et contributions\\}

Chronologiquement, nous nous sommes d'abord intéressés à la partie applicative :
nos premières contributions concernaient l'application du raisonnement
analogique à la recommandation. Ensuite, dans un second temps, nous nous sommes
intéressés à la partie théorique.

Nous avons choisi de présenter nos travaux d'une manière quelque peu
différente. Nous allons d'abord exposer les connaissances requises sur les
proportions analogiques dans les chapitres
\ref{CHAP:computational_models_of_analogical_reasoning}  et
\ref{CHAP:formal_analogical_proportions}, puis nous présenterons une partie de
nos résultats théoriques dans le chapitre  \ref{CHAP:functional_definition}.
Ensuite, nous nous attaquerons à la partie applicative dans les chapitres
\ref{CHAP:background_reco_systems} et \ref{CHAP:analogical_recommendation}.
Enfin dans le chapitre \ref{CHAP:analogy_preserving_functions},  nous
reviendrons sur des aspects théoriques.

Ce choix est en fait motivé par des raisons pédagogiques. En effet, il
apparaîtra clair que le sujet de la recommandation analogique (chapitres
\ref{CHAP:background_reco_systems} et \ref{CHAP:analogical_recommendation})
s'introduit et se comprend beaucoup plus facilement une fois que l'on a déjà
étudié le chapitre \ref{CHAP:functional_definition} qui traite de
classification analogique. De plus, bien que l'on aurait pu s'attaquer au
chapitre  \ref{CHAP:analogy_preserving_functions} directement après le chapitre
\ref{CHAP:functional_definition}, il sera d'autant plus intéressant d'étudier
le chapitre  \ref{CHAP:analogy_preserving_functions} à la lumière des
conclusions des chapitres \ref{CHAP:background_reco_systems} et
\ref{CHAP:analogical_recommendation}.

Ce document est organisé comme suit.\\

Le premier chapitre donnera les prérequis nécessaires sur différents modèles de
raisonnement par analogie, en insistant particulièrement sur les modèles
permettant l'inférence. Il apparaîtra clair que ces modèles sont, pour la
plupart, motivés par des recherches en sciences cognitives ou psychologiques,
ce qui les distingue de notre principal outil : les proportions analogiques.\\

Les proportions analogiques sont l'objet du deuxième chapitre. Nous donnerons
leurs définitions dans différents domaines algébriques, en insistant sur les
deux proportions que nous utiliseront le plus : la proportion arithmétique et
la proportion booléenne. De plus, nous essaierons d'aborder ces proportions
d'un point de vue géométrique afin d'en mieux saisir les différentes
caractéristiques. Nous étudierons aussi un problème de classification
\textit{jouet}, ce qui nous permettra d'introduire la notion de résolution
d'équation analogique, ainsi que le principe d'inférence analogique qui est
sous-jacent à toutes  nos recherches. Dans un contexte de classification, ce
principe stipule que si $a$ est à $b$ ce que $c$ est à $d$, alors on devrait
aussi avoir que $f(a)$ est à $f(b)$ comme $f(c)$ est à $f(d)$, où $f(x)$ est la
fonction qui détermine la classe de l'élément $x$. Ce principe est évidemment
non correct, dans le sens où la conclusion (les classes sont en proportion)
n'est pas logiquement déductible de la prémisse (les éléments sont en
proportion). Comme nous le verrons, il reste utilisable dans des tâches de
classification, ce qui est l'objet du troisième chapitre.\\


L'étude des classifieurs analogiques est en effet le sujet du chapitre
\ref{CHAP:functional_definition}. Ce chapitre correspond à une version étendue
d'un papier publié à l'ECAI \cite{HugPraRicSerECAI16}. Nous commencerons ici à
nous attaquer à un des deux thèmes principaux de cette thèse : identifier des
propriétés théoriques des classifieurs analogiques. Notre première contribution
sera de proposer une définition fonctionnelle des classifieurs analogiques, qui
unifie les deux approches préexistantes déjà mentionnées (\cite{StrYvoCNLL05}
et \cite{BayMicDelIJCAI07}). Cette nouvelle définition nous permettra de
dériver des résultats relatifs à la VC-dimension des classifieurs, ainsi que
leur taux d'erreur. Cette définition fonctionnelle permettra aussi de mettre
clairement en lumière les liens qui relient les classifieurs de type
plus-proches-voisins ($k$-NN) et les classifieurs analogiques. En effet, on
montrera que la classification analogique peut se concevoir comme la
successions de deux étapes. Dans un premier temps, on étend l'ensemble
d'apprentissage  en ajoutant de nouveaux exemples, puis on utilise ensuite un
classifieur $k$-NN sur cet ensemble étendu (appelé extension analogique). On
remarquera que les deux processus cognitifs mentionnés précédemment au
sujet des proportions analogiques (créativité et inférence) sont ici intimement
liés. De ces résultats, une question naturelle se pose : comment assurer que
l'extension analogique soit saine, c'est à dire comment être certain que les
exemples générés ont été associés à la bonne classe ? Cette question sera
résolue dans le chapitre \ref{CHAP:analogy_preserving_functions}, mais les deux
chapitres suivant s'intéresseront à l'application du raisonnement par analogie
au problème de la recommandation.\\


Le chapitre  \ref{CHAP:background_reco_systems} sera dédié aux prérequis
nécessaires sur les systèmes de recommandation. Nous décrierons brièvement les
trois principales familles de systèmes de recommandation, à savoir les
techniques \textit{content-based}, le filtrage collaboratif, et les techniques
\textit{knowledge-based}. Nous décrirons aussi formellement le problème que
l'on se propose d'étudier, qui est celui de la prédiction de notes : étant
donné un ensemble de notes données par des utilisateurs pour des items, le but
est de prédire d'autres notes qui ne sont pas dand la base de données. Les
différentes mesures qui permettent d'évaluer la qualité d'un système de
recommandation seront présentées. Enfin, comme nos algorithmes seront de type
\textit{filtrage collaboratif}, nous décrirons en détail deux techniques de
prédiction collaborative qui nous serviront à comparer les performances de nos
algorithmes : les méthodes de voisinage, et les méthodes fondées sur la
factorisation de matrice.\\


Dans le chapitre \ref{CHAP:analogical_recommendation}, nous présenterons nos
contributions au problème de la recommandation. Nous décrirons d'abord nos
première investigations, qui sont une adaptation directe des classifieurs
analogiques décrits dans le chapitre \ref{CHAP:functional_definition}. Ces
recherches ont fait l'objet d'un papier publié à ISMIS \cite{HugPraRicISMIS15}.
Il apparaîtra que ces algorithmes, bien que proposant une précision semblable à
celle des méthodes par voisinage, se révèlent d'une complexité calculatoire
telle qu'il n'est pas raisonnable de les envisager comme des candidats
potentiels à une implémentation dans un système effectif. Fort de ce constat,
nous avons envisagé une autre version de la recommandation analogique, cette fois
fondée sur le fait que les utilisateurs ont tous une différente interprétation
de l'échelle de note. Les algorithmes qui en découleront proposeront une
complexité algorithmique tout à fait raisonnable et sont l'objet d'un papier
Fuzz-IEEE  \cite{HugPraRicSerFuzzIEEE16}. Enfin, nous nous intéresseront au
problème de la fouille de proportions analogique dans une base de données
partiellement renseignée, telle que celles dont on dispose lorsque l'on
s'attaque à la prédiction de notes. Les résultats obtenus nous permettront de
rétrospectivement interpréter les modestes performances de nos premiers
algorithmes de recommandation analogique : même si cela semble tristement
banal, il apparaîtra qu'il n'existait probablement tout simplement pas de
bonnes analogiques dans les bases de données utilisées.\\


Mais cette observation quelque peu terre-à-terre ne nous empêchera pas de
remettre en question le principe d'inférence analogique même. Dans le dernier
chapitre, nous reviendrons à nos considérations théoriques, et nous proposeront
un critère qui permet d'appliquer ce principe d'inférence de manière correcte.
Nous avons brièvement décrit ce principe, qui stipule que si quatre élément $a,
b, c, d$ sont en proportion, alors leurs classes $f(a), f(b), f(c), f(d)$
doivent aussi être en proportion. Nous donneront une caractérisation complète
des fonctions $f$ telles que lorsque quatre éléments sont en proportion, alors
leurs images par $f$ le sont aussi. Ces fonctions permettent d'utiliser le
principe d'inférence analogique de manière sûre, et assurent une extension
analogique saine (c'est à dire sans erreur). Nous les appelleront les fonctions
qui préservent l'analogie, et nous montrerons qu'elles correspondent aux
fonctions affines dans les domaines booléens et réels. Ces travaux ont
abouti à la publication d'un papier IJCAI \cite{CouHugPraRicIJCAI17}, et
ouvrent la voie à différentes pistes de recherche.

