\paragraph{Résumé du chapitre}

D'une façon générale, une analogie établit des parallèles entre deux situations
différentes. Le raisonnement par analogie permet alors d'inférer des propriétés
sur l'une des situations, en utilisant nos connaissances de l'autre situation.
Un type particulier de raisonnement par analogie est celui lié aux proportions
analogiques, qui sont des expressions de la forme \textit{a est à b ce que c
est à d}, souvent écrit $a:b::c:d$. Là aussi, on peut parfois déduire la valeur
d'un des quatre éléments sur la base de notre connaissance des trois premiers.
Ce processus s'appelle la \textit{résolution d'équation analogique}.

Les deux premiers chapitres de ce document seront dédiés à fournir les
prérequis nécessaires concernant les modèles de raisonnement par analogie, et
en particulier ceux relatifs aux proportions analogiques. Dans ce premier
chapitre nous décrivons brièvement différentes tentatives de formaliser le
raisonnement par analogie, accordant une attention particulière aux modèles qui
peuvent s'utiliser à des fins inférentielles.

Nous avons choisi ici de faire la distinction entre deux types de modèles :
ceux qui ne font jamais référence aux proportions analogiques, et ceux qui
recourent aux proportions analogiques d'une manière ou d'une autre: les
proportions sont alors soit une composante clef du modèle (par exemple sections
\ref{SEC:solving_geometric_proportions} et \ref{SEC:rumelhart_Abrahamsen}),
soit une simple possibilité d'application (sections \ref{SEC:copycat} et
\ref{SEC:analogy_and_the_minimum_decsription_length_principle}).

Ces différentes propositions sont, en grande partie, inspirées par des travaux
provenant des sciences cognitives et psychologiques. Ce n'est qu'assez
récemment que certains auteurs, partant des premiers travaux de Lepage
\cite{Lep04}, ont proposé divers modèles formels permettant l'usage des
proportions analogiques au sein de structures algébriques. Ces
proportions analogiques sont le principal objet de nos travaux, et nous les
décrivons en détail dans le prochain chapitre.
