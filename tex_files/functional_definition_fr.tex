\paragraph{Résumé du chapitre}
Ce chapitre s'intéresse au problème de la classification analogique, et décrit
nos contributions à ce sujet.

Une première forme de classifieur analogique a été proposée dans les travaux
de Stroppa et Yvon  \cite{StrYvoCNLL05} (classifieurs conservatifs). Une autre
forme de classifieur analogique est due aux travaux de Bayoudh, Delhay et
Miclet \cite{MicBayDelJAIR08, BayMicDelIJCAI07} (classifieurs étendus).
En pratique, ces deux types de classifieurs ont des implémentations
relativement distinctes (même s'ils reposent sur les mêmes fondements
théoriques).

Notre première contribution est de proposer une définition fonctionnelle des
classifieurs analogiques, qui a la particularité d'unifier ces deux approches
préexistantes. Cette définition nous permet de dériver diverses propriétés
théoriques des classifieurs analogiques telles que leur VC-dimension (infinie),
et leur taux d'erreur qui s'exprime en fonction de celui d'un classifieur de
type plus-proches-voisins ($k$-NN). En effet, un des autres apports majeurs de
cette définition fonctionnelle, est de révéler les liens étroits qui existent
entre les classifieurs analogiques et les classifieurs $k$-NN. Aussi, on peut
considérer le processus de classification comme la succession des deux étapes
suivantes :

\begin{itemize}
  \item D'abord, l'ensemble d'apprentissage est étendu pour devenir ce que l'on
    appelle \textit{l'extension analogique}.
  \item Ensuite, un classifieur $k$-NN est utilisé sur cette extension
    analogique, qui fait office d'ensemble d'apprentissage étendu (et
    potentiellement bruité).
\end{itemize}

Naturellement, le taux d'erreur du classifieur analogique dépend de la qualité
de l'extension. Dans le chapitre \ref{CHAP:analogy_preserving_functions}, nous
décrirons dans quelles conditions cette extension est non-bruitée.


Jusqu'à présent, l'utilisation des proportions analogiques sur des problèmes
d'apprentissage était principalement cantonnée à des applications artificielles
(domaine booléen par exemple). Un des deux enjeux de cette thèse était
d'évaluer la pertinence de l'usage des proportions analogiques sur des
problèmes plus concrets. Cet axe de recherche est l'objet des deux chapitres
suivants, où nous nous intéresserons au problème de la recommandation.
