Les deux principaux objectifs de cette thèse étaient les suivants : exhiber des
propriétés théoriques des classifieurs analogiques, et appliquer le
raisonnement analogique au problème de la recommandation. Commençons par
rappeler succinctement les principaux résultats et contributions apportées dans
ce document (on ne suivra pas nécessairement l'ordre des chapitres).

\paragraph{Contributions\\}

Le premier chapitre était dévoué aux prérequis nécessaires sur les modèles de
raisonnement analogique existant, en accordant une attention particulière aux
modèles qui permettent l'implémentation de programmes informatiques. Dans le
deuxième chapitre, nous avons décrit en détail différentes définitions de
proportions analogiques, en insistant particulièrement sur les proportions
booléennes et arithmétiques. Nous avons essayé d'apporter au lecteurs quelques
intuitions géométriques sur les propriétés de ces proportions. Enfin, nous
avons considéré un problème de classification booléenne basique, qui nous a
permis d'introduire le processus de résolution d'équation analogique, ainsi que
le principe d'inférence analogique. Ce principe stipule que si l'on a
$a:b::c:d$, alors on devrait aussi avoir $f(a):f(b)::f(c):f(d)$, où $f(x)$
détermine la classe de $x$. Lorsque $f(d)$ est inconnu, on peut l'inférer à
partir des valeurs de $f(a), f(b)$ et $f(c)$ ce qui nous permet d'appliquer ce
type d'inférence à des tâches de classification ou des problèmes plus généraux
encore comme celui de la prédiction de notes.\\

La recommandation analogique fut l'objet des chapitres
\ref{CHAP:background_reco_systems} et \ref{CHAP:analogical_recommendation}.
Chapitre \ref{CHAP:background_reco_systems} était un chapitre de prérequis, où
nous avons clairement défini le problème de la prédiction de notes, et décrit
deux familles d'algorithmes (les méthodes par voisinage et par factorisation de
matrice) qui nous servent de base pour comparer nos algorithmes.

Nous avons ensuite développé dans le chapitre
\ref{CHAP:analogical_recommendation} un premier algorithme pour la prédiction
de notes  \cite{HugPraRicISMIS15}, fondé sur l'idée que si quatre utilisateurs
sont en proportion (c.a.d. que leurs notes respectives sont en proportion
géométrique), alors cela doit aussi être vrai pour un item que l'un des
utilisateurs n'a pas encore noté. Cette approche est directement inspirée des
travaux en classification analogique. Nos expériences ont montré que les
performances de ces algorithmes sont proches de celles des techniques par
voisinage, mais leur complexité cubique conduisent à des temps de calculs
prohibitifs.

Cela nous a conduit à considérer une autre approche pour la recommandation
analogique, qui ne repose pas sur la recherche de triplets d'utilisateurs.
Cette approche \cite{HugPraRicSerFuzzIEEE16} qui repose sur la notion de
``\textit{clones}'', prend en compte le fait que chaque utilisateur a une
appréciation bien particulière de l'échelle de note qui est utilisée. Nos
résultats montrent que le concept de clone est très important et qu'il est
primordial de l'intégrer d'une manière ou d'une autre dans un algorithme de
prédiction de notes. Il est cependant nécessaire de noter que ce type de biais
de les notes avait déjà été considéré dans de précédents travaux. Nous noterons
cependant que nos travaux sur les systèmes de recommandation nous ont conduit à
développer une bibliothèque Python \cite{Surprise} qui permet de facilement
tester et développer des algorithmes de prédiction de note.

Enfin, nous avons proposé un algorithme pour la fouille de proportions
analogiques dans des bases de données incomplètes \cite{HugPraRicSerLFA16}, en
nous inspirant fortement des travaux en fouille de règles d'associations. Nous
avons utilisé cet algorithme pour extraire des proportions analogiques entre
films dans la base de données MovieLens, qui était celle que nous utilisions
pour nos algorithmes de prédiction de note. Les résultats suggèrent qu'en fait,
les analogiques potentielles restent d'un intérêt relativement faible, en ceci
qu'elle mettaient généralement en jeu des items soit également appréciés, soit
également désapprouvés. En d'autre termes, les notes relatives aux différents
utilisateurs pour les quatre films d'une même proportion étaient souvent les
mêmes. Ceci nous permet de rétrospectivement interpréter les résultats modestes
de notre premier algorithme pour la prédiction de note, car cela suggère que
finalement, l'algorithme ne pouvait trouver que trop peu de bonnes proportions
pour produire des estimations de qualité.\\


Dans le chapitre \ref{CHAP:functional_definition}, nous avons décrit nos
contributions au problème de la classification analogique
\ref{CHAP:functional_definition}. Notre première contribution a été de proposer
une définition fonctionnelle des classifieurs analogiques, qui a a
particularité d'unifier les approches préexistantes. A partir de cette
définition, nous avons été capables de dériver différentes propriétés
théoriques. En particulier, nous avons montré que la VC-dimension des
classifieurs analogiques était infinie, et que leur taux d'erreur est
intimement lié à celui des algorithmes $k$-NN. En fait, notre définition
fonctionnelle établie clairement le lien qui relie la classification analogique
et la classification par plus-proches-voisins. Nous avons montré en effet que
la classification se traduit en deux étapes. La première étape consiste à
étendre l'ensemble d'apprentissage en ajoutant de nouveaux exemples dont la
classe est estimée en utilisant le principe d'inférence analogique. Ensuite,
sur la base de cette extension analogique, un algorithme $k$-NN est utilisé
pour classer le reste des éléments qui ne sont pas dans l'extension. Ce nouveau
point de vue est assez intéressant car il fait clairement intervenir deux des
principales caractéristiques des proportions analogiques, à savoir la
créativité et l'inférence.

Dans le chapitre \ref{CHAP:analogy_preserving_functions}, nous nous sommes
intéressés à une question qui découle naturellement des conclusions du chapitre
\ref{CHAP:functional_definition}: comment assurer que l'extension analogique
soit saine ? En d'autres termes, existe-t-il un critère qui nous permette
d'être sûr que les éléments de l'extension analogique sont correctement
classés ? Nous avons été capables de répondre à cette question  pour les
domaines booléens : l'extension analogique est parfaitement saine si (et
seulement si) la fonction booléenne $f$ qui détermine les labels est une
fonction affine. Nous avons étendu ces résultats aux domaines réels, et au cas
où les attributs ont des valeurs nominales. Cette série de travaux ouvre la
porte à divers pistes de recherches futures, que nous mentionnons maintenant.

\paragraph{Axes de recherches futures\\}

Dans la section \ref{SEC:approximate_ap_functions}, nous avons présenté le
concept de fonction approximativement AP (fonction qui préserve
approximativement l'analogie). En effet en pratique, il n'y a aucun moyen de
savoir si la fonction cible $f$ est vraiment affine. Ainsi, il est d'un intérêt
crucial d'obtenir des garanties statistiques relatives à la qualité de
l'extension analogique lorsque la fonction $f$ s'éloigne de l'ensemble des
fonctions affines. Un résultat comme le suivant serait très intéressant :

$$P(\omegasf \geq \eta) > 1 - \delta,$$
où $\eta \in [0, 1]$ et $\delta \in [0, \frac{1}{2}]$ sont des fonctions 
qui dépendent de $\varepsilon$ et de $\mid S \mid$. La valeur $\varepsilon$
indique que $f$ est $\varepsilon$-proche de l'ensemble des fonctions affines.
Pour le moment, nous nous intéressons à un problème un peu plus simple. On se
propose en effet de chercher une borne $\delta$ telle que :

$$P(\omegasf < 1) < \delta,$$
où $\delta$ toujours fonction de $\varepsilon$.  En d'autres termes, on
cherche à borner la probabilité de l'événement ``\textit{$\esfs$ contient des
erreurs}''\\


Nous avons aussi remarqué lors de nos expérimentations que même lorsqu'une
fonction ne préserve pas du tout l'analogie, la qualité de l'extension
analogique peut tout de même être très bonne. Cela est du au fait que la
procédure de vote majoritaire permet parfois de contrebalancer les erreurs de
certains predicteurs. Il est clair qu'une compréhension profonde du rôle de ce
vote majoritaire reste une étape clef dans l'étude des classifieurs
analogiques.

Un autre axe de recherche possible serait de tirer partie du fait que toute
fonction booléenne est en fait affine par morceaux. Si l'on peut être sûr que
quatre éléments sont dans un domaine où $f$ est affine, alors on peut appliquer
de manière correct le principe d'inférence analogique. Pour l'instant, nous
n'avons mené que des expérimentations très préliminaires  à ce sujet.\\

Enfin, nous terminerons cette discussion en évoquant l'apprentissage par
transfert (\textit{transfer learning}). L'apprentissage par transfert est un
sujet assez populaire de nos jours dans la communauté de l'apprentissage
artificiel, dont le but est de transférer des connaissances d'un problème
source $S$ pour les appliquer à un problème moins bien connu $T$.
Indubitablement, l'essence même de l'analogie est d'opérer un transfert de
connaissance. L'usage des proportions analogiques à un tel domaine
d'application reste encore entièrement à explorer.
