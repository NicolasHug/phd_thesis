Le raisonnement par analogie est largement considéré comme une caractéristique
clef de l'intelligence humaine. Il permet d'arriver à de potentielles
conclusions pour de nouvelles situations, en établissant des liens entre des
domaines en apparence très différents. Un exemple connu  est celui du modèle de
l'atome de Rutherford-Bohr, où les électrons gravitent autour du noyau, comme
les planètes du système solaire gravitent autour du soleil. Ce type de
raisonnement a fait l'objet de nombreuses études de la part des psychologues et
des philosophes, mais il intéresse aussi la communauté de l'intelligence
artificielle.

Ce travail s'intéresse à un mode particulier de raisonnement par analogie, à
savoir le raisonnement par proportions analogiques. Une proportion analogique
est une proposition de la forme \textit{a est à b comme c est à d}, et exprime
le fait que $a$ diffère de $b$ comme $c$ diffère de $d$. On pourrait par
exemple affirmer que la \textit{France est à Paris ce que l'Allemagne est à
Berlin}, ou encore que \textit{les électrons sont au noyau ce que les planètes
sont au soleil}. Dans certains cas, lorsque l'élément $d$ n'est pas connu, on
pourra l'inférer à partir des trois autres éléments $a$, $b$, et $c$. Il semble
en effet naturel qu'un enfant doté  de connaissances géographiques basiques
pourrait répondre à la question \textit{La France est Paris ce que l'Allemagne
est à quoi ?} Et même dans l'éventualité où l'enfant ne connaîtrait pas le nom
de la ville \textit{Berlin}, il pourrait tout de même s'imaginer une
hypothétique ville au nom inconnu, mais qui serait la capitale de l'Allemagne.
L'utilisation des proportions analogiques nous permet ici de toucher du doigt
deux processus cognitifs clef : l'inférence, et la créativité.

En tant qu'outil permettant une forme d'inférence inductive, le raisonnement
par analogie a déjà été envisagé pour des problèmes d'apprentissage artificiel.
On pourra citer par exemple \cite{StrYvoCNLL05} pour une application en
linguistique, ainsi que \cite{BayMicDelIJCAI07} pour des problèmes de
classification dans des domaines booléens. Nos recherches s'inscrivent
dans la lignée de ces deux précédents travaux, et le but de cette thèse est
double. Le premier sujet de recherche est d'appliquer le raisonnement
analogique à des problèmes concrets, et le second est d'étudier les classifieurs
analogiques d'un point de vue théorique, afin d'en exhiber des propriété
théoriques intéressantes.

L'application du raisonnement analogique à des problèmes concrets est motivée
par le fait que les précédentes recherches ont principalement été menées d'un
point de vue expérimental, dans des environnements assez artificiels. A notre
connaissance, aucun travail n'avait été mené pour évaluer la pertinence des
classifieurs analogiques sur des tâches concrètes, où les critères de succès
sont souvent différents. Nous avons ici choisi d'appliquer le raisonnement
analogique au problème de la recommandation, et plus particulièrement à celui
de la prédiction de notes.

D'autre part, nos recherches théoriques sont motivées par le fait que
jusqu'alors, les classifieurs analogiques n'étaient connus que via leurs
descriptions algorithmiques. On avait une connaissance claire de comment
calculer la sortie d'un classifieur analogique, mais la nature de ce qui était
effectivement calculé restait encore assez floue. On manquait alors d'outils
pour caractériser les forces et les faiblesses de tels classifieurs.

\paragraph{Annonce du plan et contributions\\}

Chronologiquement, nous nous sommes d'abord intéressés à la partie applicative :
nos premières contributions concernaient l'application du raisonnement
analogique à la recommandation. Ensuite, dans un second temps, nous nous sommes
intéressés à la partie théorique.

Nous avons choisi de présenter nos travaux d'une manière quelque peu
différente. Nous allons d'abord exposer les connaissances requises sur les
proportions analogiques dans les chapitres
\ref{CHAP:computational_models_of_analogical_reasoning}  et
\ref{CHAP:formal_analogical_proportions}, puis nous présenterons une partie de
nos résultats théoriques dans le chapitre  \ref{CHAP:functional_definition}.
Ensuite, nous nous attaquerons à la partie applicative dans les chapitres
\ref{CHAP:background_reco_systems} et \ref{CHAP:analogical_recommendation}.
Enfin dans le chapitre \ref{CHAP:analogy_preserving_functions},  nous
reviendrons sur des aspects théoriques.

Ce choix est en fait motivé par des raisons pédagogiques. En effet, il
apparaîtra clair que le sujet de la recommandation analogique (chapitres
\ref{CHAP:background_reco_systems} et \ref{CHAP:analogical_recommendation})
s'introduit et se comprend beaucoup plus facilement une fois que l'on a déjà
étudié le chapitre \ref{CHAP:functional_definition} qui traite de
classification analogique. De plus, bien que l'on aurait pu s'attaquer au
chapitre  \ref{CHAP:analogy_preserving_functions} directement après le chapitre
\ref{CHAP:functional_definition}, il sera d'autant plus intéressant d'étudier
le chapitre  \ref{CHAP:analogy_preserving_functions} à la lumière des
conclusions des chapitres \ref{CHAP:background_reco_systems} et
\ref{CHAP:analogical_recommendation}.

Ce document est organisé comme suit.\\

Le premier chapitre donnera les prérequis nécessaires sur différents modèles de
raisonnement par analogie, en insistant particulièrement sur les modèles
permettant l'inférence. Il apparaîtra clair que ces modèles sont, pour la
plupart, motivés par des recherches en sciences cognitives ou psychologiques,
ce qui les distingue de notre principal outil : les proportions analogiques.\\

Les proportions analogiques sont l'objet du deuxième chapitre. Nous donnerons
leurs définitions dans différents domaines algébriques, en insistant sur les
deux proportions que nous utiliseront le plus : la proportion arithmétique et
la proportion booléenne. De plus, nous essaierons d'aborder ces proportions
d'un point de vue géométrique afin d'en mieux saisir les différentes
caractéristiques. Nous étudierons aussi un problème de classification
\textit{jouet}, ce qui nous permettra d'introduire la notion de résolution
d'équation analogique, ainsi que le principe d'inférence analogique qui est
sous-jacent à toutes  nos recherches. Dans un contexte de classification, ce
principe stipule que si $a$ est à $b$ ce que $c$ est à $d$, alors on devrait
aussi avoir que $f(a)$ est à $f(b)$ comme $f(c)$ est à $f(d)$, où $f(x)$ est la
fonction qui détermine la classe de l'élément $x$. Ce principe est évidemment
non correct, dans le sens où la conclusion (les classes sont en proportion)
n'est pas logiquement déductible de la prémisse (les éléments sont en
proportion). Comme nous le verrons, il reste utilisable dans des tâches de
classification, ce qui est l'objet du troisième chapitre.\\


L'étude des classifieurs analogiques est en effet le sujet du chapitre
\ref{CHAP:functional_definition}. Ce chapitre correspond à une version étendue
d'un papier publié à l'ECAI \cite{HugPraRicSerECAI16}. Nous commencerons ici à
nous attaquer à un des deux thèmes principaux de cette thèse : identifier des
propriétés théoriques des classifieurs analogiques. Notre première contribution
sera de proposer une définition fonctionnelle des classifieurs analogiques, qui
unifie les deux approches préexistantes déjà mentionnées (\cite{StrYvoCNLL05}
et \cite{BayMicDelIJCAI07}). Cette nouvelle définition nous permettra de
dériver des résultats relatifs à la VC-dimension des classifieurs, ainsi que
leur taux d'erreur. Cette définition fonctionnelle permettra aussi de mettre
clairement en lumière les liens qui relient les classifieurs de type
plus-proches-voisins ($k$-NN) et les classifieurs analogiques. En effet, on
montrera que la classification analogique peut se concevoir comme la
successions de deux étapes. Dans un premier temps, on étend l'ensemble
d'apprentissage  en ajoutant de nouveaux exemples, puis on utilise ensuite un
classifieur $k$-NN sur cet ensemble étendu (appelé extension analogique). On
remarquera que les deux processus cognitifs mentionnés précédemment au
sujet des proportions analogiques (créativité et inférence) sont ici intimement
liés. De ces résultats, une question naturelle se pose : comment assurer que
l'extension analogique soit saine, c'est à dire comment être certain que les
exemples générés ont été associés à la bonne classe ? Cette question sera
résolue dans le chapitre \ref{CHAP:analogy_preserving_functions}, mais les deux
chapitres suivant s'intéresseront à l'application du raisonnement par analogie
au problème de la recommandation.\\


Le chapitre  \ref{CHAP:background_reco_systems} sera dédié aux prérequis
nécessaires sur les systèmes de recommandation. Nous décrierons brièvement les
trois principales familles de systèmes de recommandation, à savoir les
techniques \textit{content-based}, le filtrage collaboratif, et les techniques
\textit{knowledge-based}. Nous décrirons aussi formellement le problème que
l'on se propose d'étudier, qui est celui de la prédiction de notes : étant
donné un ensemble de notes données par des utilisateurs pour des items, le but
est de prédire d'autres notes qui ne sont pas dans la base de données. Les
différentes mesures qui permettent d'évaluer la qualité d'un système de
recommandation seront présentées. Enfin, comme nos algorithmes seront de type
\textit{filtrage collaboratif}, nous décrirons en détail deux techniques de
prédiction collaborative qui nous serviront à comparer les performances de nos
algorithmes : les méthodes de voisinage, et les méthodes fondées sur la
factorisation de matrice.\\


Dans le chapitre \ref{CHAP:analogical_recommendation}, nous présenterons nos
contributions au problème de la recommandation. Nous décrirons d'abord nos
première investigations, qui sont une adaptation directe des classifieurs
analogiques décrits dans le chapitre \ref{CHAP:functional_definition}. Ces
recherches ont fait l'objet d'un papier publié à ISMIS \cite{HugPraRicISMIS15}.
Il apparaîtra que ces algorithmes, bien que proposant une précision semblable à
celle des méthodes par voisinage, se révèlent d'une complexité calculatoire
telle qu'il n'est pas raisonnable de les envisager comme des candidats
potentiels à une implémentation dans un système effectif. Fort de ce constat,
nous avons envisagé une autre version de la recommandation analogique, cette fois
fondée sur le fait que les utilisateurs ont tous une différente interprétation
de l'échelle de note. Les algorithmes qui en découleront proposeront une
complexité algorithmique tout à fait raisonnable et sont l'objet d'un papier
Fuzz-IEEE  \cite{HugPraRicSerFuzzIEEE16}. Enfin, nous nous intéresseront au
problème de la fouille de proportions analogique dans une base de données
partiellement renseignée, telle que celles dont on dispose lorsque l'on
s'attaque à la prédiction de notes. Les résultats obtenus nous permettront de
rétrospectivement interpréter les modestes performances de nos premiers
algorithmes de recommandation analogique : même si cela semble tristement
banal, il apparaîtra qu'il n'existait probablement tout simplement pas de
bonnes analogiques dans les bases de données utilisées.\\


Mais cette observation quelque peu terre-à-terre ne nous empêchera pas de
remettre en question le principe d'inférence analogique même. Dans le dernier
chapitre, nous reviendrons à nos considérations théoriques, et nous proposeront
un critère qui permet d'appliquer ce principe d'inférence de manière correcte.
Nous avons brièvement décrit ce principe, qui stipule que si quatre élément $a,
b, c, d$ sont en proportion, alors leurs classes $f(a), f(b), f(c), f(d)$
doivent aussi être en proportion. Nous donneront une caractérisation complète
des fonctions $f$ telles que lorsque quatre éléments sont en proportion, alors
leurs images par $f$ le sont aussi. Ces fonctions permettent d'utiliser le
principe d'inférence analogique de manière sûre, et assurent une extension
analogique saine (c'est à dire sans erreur). Nous les appelleront les fonctions
qui préservent l'analogie, et nous montrerons qu'elles correspondent aux
fonctions affines dans les domaines booléens et réels. Ces travaux ont
abouti à la publication d'un papier IJCAI \cite{CouHugPraRicIJCAI17}, et
ouvrent la voie à différentes pistes de recherche.
