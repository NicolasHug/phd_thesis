\paragraph{Résumé du chapitre}

Nous avons introduit le principe d'inférence analogique au chapitre
\ref{CHAP:formal_analogical_proportions}. Ce principe stipule que si quatre
éléments sont en proportion, alors cela doit aussi être vrai pour leurs classes
(ou n'importe quel autre attribut). C'est évidemment un principe non-correct,
puisque la conclusion ne se déduit pas logiquement de la prémisse. Néanmoins,
on a pu observer au chapitre \ref{CHAP:functional_definition}, ainsi que dans
\cite{BayMicDelIJCAI07}, que les classifieurs analogiques semblaient afficher
des résultats prometteurs (au moins sur certains problèmes). Le but de ce
chapitre est d'identifier clairement et de caractériser les types de problèmes
sur lesquels le principe d'inférence analogique est correct. En d'autres
termes, nous proposons de caractériser les fonctions booléennes $f$ telles que
l'inférence suivante soit correcte:

$$
\inferrule{\mathbf{a} : \mathbf{b} :: \mathbf{c} : \mathbf{d}}{ f(\mathbf{a}) :
f(\mathbf{b}) :: f(\mathbf{c}) : f(\mathbf{d})}
$$

Il est en effet assez facile de construire des problèmes simples où
l'application du principe d'inférence analogique mène a des résultats
désastreux. Considérons par exemple celui de la figure \ref{FIG:classif_in_R2}:
l'ensemble d'apprentissage $S$ (à gauche) est linéairement séparable par une
droite verticale. Cependant, lorsqu'il est étendu pour former son extension
analogique (à droite), on constate que l'espace entier est criblé de points
appartenant à l'une ou l'autre des deux classes. Utiliser cette extension
analogique comme ensemble d'apprentissage mènera nécessairement à de très
mauvais résultats.

Les fonctions $f$ qui sont compatibles avec le principe d'inférence analogiques
sont les fonctions dites \textit{qui préservent l'analogique} (fonctions AP),
c'est à dire des fonctions qui satisfont $f(\mathbf{a}) :f(\mathbf{b})::
f(\mathbf{c}): f(\mathbf{d})$  dès lors que $a:b::c:d$ est vrai (en fait, une
définition moins restrictive et plus utile en pratique consiste à ajouter la
prémisse selon laquelle $f(\mathbf{a}) :f(\mathbf{b}):: f(\mathbf{c}):y$ est
soluble).


Dans un contexte booléen, nous avons prouvé que les fonctions AP sont les
fonctions affines. Notre preuve repose sur l'utilisation d'outils algébriques
tels que la forme normale algébrique des fonctions booléennes. Tout fonction
booléenne peut être exprimée sous la forme d'un polynôme, et notre preuve
consiste d'abord  à montrer que les polynômes de degré $1$ (c'est à dire les
fonctions affines) sont AP, puis à montrer que tout polynôme de degré supérieur
ou égal à $2$ n'est pas AP.

Évidemment, dans des problèmes pratiques les fonctions $f$ rencontrée sont très
variées et ne se cantonnent pas à l'ensemble des fonctions affines. Aussi,
nous avons expérimentalement étudié la qualité de l'extension analogique
lorsque la fonction $f$ s'éloigne de l'ensemble des fonctions AP de différentes
manières. D'abord, nous avons observé les variations de la qualité de
l'extension en fonction de la distance $\varepsilon$ de $f$ à l'ensemble des
fonctions affines. Ensuite, nous avons observé que même lorsqu'une fonction
n'est vraiment pas AP (car une partie des prédicteurs se trompent
régulièrement), il est toujours possible d'obtenir une très bonne qualité de
l'extension analogique grâce à la procédure de votre majoritaire qui permet de
compenser les erreurs des mauvais prédicteurs.

Enfin, nous avons partiellement étendu nos résultats au cas où la fonction $f$
est une fonction réelles, et au cas où les attributs ne sont plus booléens mais
plutôt nominaux.
