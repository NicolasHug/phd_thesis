\chapter*{Abstract}

\initial{A}nalogical reasoning is recognized as a core component of human
intelligence. It has been extensively studied from philosophical and
psychological viewpoints, but recent works also address the modeling of
analogical reasoning for computational purposes, particularly focused on
analogical proportions. We are interested here in the use of analogical
proportions for making predictions, in a machine learning context.

In recent works, analogy-based classifiers have given noticeable performances,
in particular by performing extremely well on some sets of artificial problems
where other traditional methods would fail. Starting from this empirical
observation, the goal of this thesis is twofold. The first topic of research is
to exhibit meaningful theoretical properties of analogical classifiers, which
were yet only empirically studied.  The second topic is to assess the relevance
of analogical learners on real-world, practical application problems.

The field of application that was chosen for assessing the suitability of
analogical classifiers in real-world setting is the topic of recommender
systems. A common reproach addressed towards recommender systems is that they
often lack of novelty and diversity in their recommendations. As a way of
establishing links between seemingly unrelated objects, analogy was thought as
a way to overcome this issue. Experiments here show that while offering
sometimes similar accuracy performances to those of basic classical approaches,
analogical classifiers still suffer from their algorithmic complexity.

On the theoretical side, a key contribution of this thesis is to provide a
functional definition of analogical classifiers, that unifies the various
pre-existing approaches. So far, only algorithmic definitions were known, making
it impossible to lead a thorough theoretical study. From this functional
definition, we clearly identified the links between our approach and that of
the nearest neighbors learners, in terms of process and in terms of accuracy.
We were also able to identify a criterion that ensures a safe application of our
analogical inference principle, which allows us to characterize analogical
reasoning as some sort of linear process.
\newpage
\initial{L}e raisonnement par analogie est reconnu comme une des principales
caractéristiques de l'intelligence humaine. En tant que tel, il a pendant
longtemps été étudié par les philosophes et les psychologues, mais de récents
travaux s'intéressent aussi à sa modélisation d'un point de vue formel à l'aide
de proportions analogiques, permettant l'implémentation de programmes
informatiques. Nous nous intéressons ici à l'utilisation des proportions
analogiques à des fins prédictives, dans un contexte d'apprentissage
artificiel.

Dans de récents travaux, les classifieurs analogiques ont montré qu'ils sont
capables d'obtenir d'excellentes performances sur certains problèmes
artificiels là où d'autres techniques traditionnelles d'apprentissage se montrent
beaucoup moins efficaces. Partant de cette observation empirique, cette thèse
s'intéresse à deux axes principaux de recherche.  Le premier sera de confronter
le raisonnement par proportion analogique à des applications pratiques, afin
d'étudier la viabilité de l'approche analogique sur des problèmes concrets.
Le second axe de recherche sera d'étudier les classifieurs analogiques d'un
point de vue théorique, car jusqu'à présent ceux-ci n'étaient connus que grâce
à leurs définitions algorithmiques. Les propriétés théoriques qui découleront
nous permettront de comprendre plus précisément leurs forces, ainsi que leurs
faiblesses.

Comme domaine d'application, nous avons choisi celui des systèmes de
recommandation. On reproche souvent à ces derniers de manquer de nouveauté ou
de surprise dans les recommandations qui sont adressées aux utilisateurs. Le
raisonnement par analogie, capable de mettre en relation des objets en
apparence différents, nous est apparu comme un outil potentiel pour répondre à
ce problème. Nos expériences montreront que les systèmes analogiques ont
tendance à produire des recommandations d'une qualité comparable à celle des
méthodes existantes, mais que leur complexité algorithmique cubique les
pénalise fortement pour prétendre à des applications pratiques où le temps de
calcul est une des contraintes principales.

Du côté théorique, une contribution majeure de cette thèse est de proposer une
définition fonctionnelle des classifieurs analogiques, qui a la particularité
d'unifier les approches préexistantes. Cette définition fonctionnelle nous
permettra de clairement identifier les liens sous-jacents entre l'approche
analogique et l'approche par $k$ plus proches voisins, tant au plan
algorithmique de haut niveau qu'au plan des propriétés théoriques (taux
d'erreur notamment). De plus, nous avons pu identifier un critère
qui rend l'application de notre principe d'inférence analogique parfaitement
certaine (c'est-à-dire sans erreurs),  exhibant ainsi les propriétés linéaires
du raisonnement par analogie.

