\chapter*{Abstract}

\initial{A}nalogical reasoning is recognized as a core component of human intelligence.
As such, it has been extensively studied from philosophical and psychological
viewpoints. We are interested here in the use of analogical reasoning for
making predictions, in a machine learning context.

In recent works, analogy-based classifiers have given noticeable performances,
in particular by performing extremely well on some sets of artificial problems
where other traditional methods would fail. Starting from this empirical
observation, the goal of this thesis is twofold. The first topic of research is
to exhibit some theoretical properties of analogical classifiers, which were
yet quite poorly known. The second topic is to assess the relevance of
analogical classifiers on real-world, practical application problems.

One of the main contribution of this thesis is to provide a functional
definition of analogical classifiers. So far, only algorithmic definition were
known, making it impossible to lead a thorough theoretical study. From this
functional definition, we identified a criteria that gives theoretical
guarantees about applying the analogical inference principle.

The field of application that was chosen for assessing the suitability of
analogical classifiers in real-world setting is the topic of recommender
systems. A common reproach addressed towards recommender systems is that they
often lack of novelty and diversity in their recommendations. As a way of
establishing links between seemingly unrelated objects, analogy was thought as
a way to overcome this issue. Experiments here show that while offering
sometimes similar accuracy performances to those of basic classical approaches,
analogical classifiers still suffer from their algorithmic complexity.
