\chapter*{Abstract}

\initial{A}nalogical reasoning is recognized as a core component of human
intelligence. It has been extensively studied from philosophical and
psychological viewpoints, but recent works also address the modeling of
analogical reasoning for computational purposes, particularly focused on
analogical proportions. We are interested here in the use of analogical
proportions for making predictions, in a machine learning context.

In recent works, analogy-based classifiers have given noticeable performances,
in particular by performing extremely well on some sets of artificial problems
where other traditional methods would fail. Starting from this empirical
observation, the goal of this thesis is twofold. The first topic of research is
to exhibit meaningful theoretical properties of analogical classifiers, which
were yet only empirically studied.  The second topic is to assess the relevance
of analogical learners on real-world, practical application problems.

The field of application that was chosen for assessing the suitability of
analogical classifiers in real-world setting is the topic of recommender
systems. A common reproach addressed towards recommender systems is that they
often lack of novelty and diversity in their recommendations. As a way of
establishing links between seemingly unrelated objects, analogy was thought as
a way to overcome this issue. Experiments here show that while offering
sometimes similar accuracy performances to those of basic classical approaches,
analogical classifiers still suffer from their algorithmic complexity.

On the theoretical side, a key contribution of this thesis is to provide a
functional definition of analogical classifiers, that unifies the various
pre-existing approaches. So far, only algorithmic definition were known, making
it impossible to lead a thorough theoretical study. From this functional
definition, we clearly identified the links between our approach and that of
the nearest neighbors learners, in terms of process and in terms of accuracy.
We were also able to identify a criteria that ensures a safe application of our
analogical inference principle.
