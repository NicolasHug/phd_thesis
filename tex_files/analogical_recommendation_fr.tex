\paragraph{Résumé du chapitre}

Ce chapitre décrit nos contributions à la recommandation analogique.

Nous avons d'abord décrit nos premières expérimentations relatives  à
l'implémentation d'un algorithme de prédiction de notes fondé sur les
proportions analogiques. Cet algorithme est une application directe du principe
d'inférence analogique, et il est très étroitement lié aux classifieurs étendus
que nous avons décrits dans le chapitre \ref{CHAP:functional_definition}.
L'idée principale derrière cet algorithme est que si quatre utilisateurs $(a,
b, c, d)$ sont en proportion vis-à-vis des items qu'ils ont conjointement notés
(en utilisant la proportion arithmétique), alors cela devrait aussi être le cas
pour d'autres items que $a, b, c$ ont noté mais que $d$ n'a pas encore noté. La
prédiction de la note de $d$ pour ces items est alors le résultat de la
résolution de plusieurs équations analogiques. Notons qu'il est parfaitement
possible de considérer des proportions entre items plutôt que des proportions
entre utilisateurs.

Nous avons comparé cet algorithme avec les deux principales approches de
filtrage collaboratif : les méthodes par voisinage et les méthodes par
factorisation de matrice. Les performances de notre algorithme sont raisonnables
du point de vue de la précision, comparativement aux méthodes par voisinage.
Cependant, leur complexité algorithmique leur est bien supérieur et les rend
inadaptés à des usages où le temps de calcul est une des contraintes
principales.

Ces observations nous on conduit à proposer un autre point de vue selon lequel
l'usage des proportions analogiques pourrait  être utilisé  à des fins de
prédiction de note. Pour cela, nous avons introduit la notions de
\textit{clones}: deux utilisateurs sont des clones s'ils ont tendance à donner
des notes \textit{sémantiquement} équivalentes. Cependant, cette équivalence
sémantique n'est pas forcément reflétée dans les notes : pour un utilisateur
$u$, une note  de $3$ peut représenter une bonne note, alors que pour
l'utilisateur $v$ cela représenterait une mauvaise note, ou juste une note
moyenne. Nous avons proposé deux algorithmes qui suivent cette idée. Nous avons
aussi présenté une contribution récente de technique par filtrage collaboratif
motivée par le fait que certains utilisateurs ont tendance à donner des notes
plus élevées que d'autres, et que certains items ont tendance à être noté avec
plus de clémence. Nous avons expliqué par quels aspects ces deux approches se
rapprochent et diffèrent l'une de l'autre. D'un point de vue de performances,
les deux méthodes restent assez similaire avec tout de même un léger avantage
pour les méthodes qui ne reposent pas sur les \textit{clones}.


Enfin, nous avons proposé un algorithme pour la fouille de proportions
analogiques dans des bases de données partiellement complètes. Cette méthode
générale s'applique parfaitement au problème de la fouille de proportions
analogiques entre utilisateurs ou entre items, dans une base de notes telle que
celle utilisée dans nos précédentes expérimentations. Notre algorithme est
inspiré de la fouille de règles d'associations et repose sur l'algorithme
Apriori pour extraire des analogiques potentielles de manière efficace. Nous
avons proposé divers fonctions de qualité permettant d'évaluer la qualité d'une
proportion, lesquels sont très proches de la dissimilarité analogique décrite
dans le chapitre \ref{CHAP:functional_definition}. Nous avons donné divers
exemples de proportions trouvées dans la base de notes, révélant des
proportions entre films. Une des principales observations est que les meilleurs
analogiques que l'on pouvait extraire de cette base ne mettaient en jeu que des
utilisateurs avec des goûts similaires : les quatre films d'une même
proportions étaient généralement également aimés, ou également rejetés. En ce
sens, on peut considérer que les quatre items d'une même proportions sont en
fait des voisins.

Ceci nous permet d'interpréter rétrospectivement les résultats modestes de nos
premières expérimentations sur la prédiction de note. nous avons vu que les
résultats de notre algorithme étaient proches de ceux des méthodes par
voisinage, ce qui s'explique maintenant par le fait que les meilleurs
proportions dont l'algorithme disposait mettaient en jeu des voisins. En
quelque sorte, notre algorithme analogique était réduit à une approche par
voisinage classique. Il est clair que dans ce cas là, l'usage de la
recommandation analogique n'en vaut as la peine puisque l'on est obligé de
payer le prix d'une complexité cubique. Il reste à noter que ces résultats ont
été obtenus sur la base de données MovieLens : on peut toujours supposer que
sur une autre base, les proportions potentielles restent significatives et
qu'alors l'usage de la recommandation analogique reste intéressant.

Pour finir, permettons nous de mentionner que nos expérimentations sur la
recommandation analogique nous ont conduit à développer un package Python
appelé Surprise \cite{Surprise}, dont les principales fonctionnalités sont les
suivantes

\begin{itemize}
  \item Les utilisateurs peuvent avoir un contrôle total sur le protocole
    expérimental mis en oeuvre. A cet effet, une attention particulière a été
    attachée à la documentation, que l'on a essayé de rendre le plus claire
    et détaillée possible.
  \item La gestion des jeux de données est rendue facile. Certaines bases de
    données classiques sont pre-intégrées, telle que les bases MovieLens.
  \item Divers algorithmes de prédiction sont implémentés, tels que les
    méthodes de voisinage, des méthodes par factorisation de matrice, et
    d'autres. D'autre part, plusieurs mesures de similarité sont disponibles
    d'office.
  \item Le création de nouveaux algorithmes de prédiction est extrêmement simple.
  \item Les procédure de validation croisée sont faciles à implémenter. Une autre
    fonctionnalité intéressante est la possibilité d'effectuer des recherches
    exhaustives sur l'espace des paramètres d'un algorithme, pour déterminer la
    combinaison de paramètre qui produit les meilleurs résultats (\textit{grid
    search}).
\end{itemize}
