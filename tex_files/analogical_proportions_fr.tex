\paragraph{Résumé du chapitre}
Ce chapitre est dédié à la présentation du principal outil utilisé dans ces
travaux: les proportions analogiques. Une proportion analogique est une
expression de la forme \textit{a est à b ce que c est à d}. Cette expression
exprime le fait que ce qui est commun à $a$ et $b$ l'est aussi à $c$ et $d$, et
ce qui diffère entre $a$ et $b$ diffère aussi entre $c$ et $d$. On pourra
trouver de nombreux exemples de proportion analogique, par exemple ``\textit{le
veau est à la vache comme le poulain est à la jument}''.

Selon Aristote, une proportion analogique $A$ est une relation quaternaire qui
satisfait les trois axiomes suivant, pour tout élément $a, b, c, d$:

\begin{enumerate}
\item $A(a,b,a,b)$ est toujours vrai (réflexivité)
\item $A(a,b,c,d) \implies A(c,d,a,b)$ (symétrie)
\item $A(a,b,c,d) \implies A(a,c,b,d)$ (permutation centrale)
\end{enumerate}

Lorsqu'il n'y a pas d'ambiguité sur $A$ et son domaine, on se permettra
d'utiliser la notation infixe $a:b::c:d$. Considérant à nouveau notre exemple
fermier, l'axiome de symétrie indique si le veau  $(a)$ est à la vache $(b)$
ce que le poulain $(c)$ est à la jument $(d)$, alors le poulain $(c)$ est  à la
jument $(d)$ ce que le veau $(a)$ est à la vache $(b)$, ce qui semble tout à
fait naturel. L'axiome de permutation centre indique qu'une autre conséquence
est que le veau $(a)$ est au poulain $(c)$ ce que la vache $(b)$ est à la
jument $(d)$.

Dans la première section, nous avons formellement défini les proportions
analogiques dans divers domaines algébriques, partant de la définition générale
dans les semi-groupes pour arriver aux proportions booléennes. Nous avons aussi
brièvement présenté une extension multi-valuée de la proportion booléenne qui
nous sera utile bien plus tard au chapitre
\ref{CHAP:analogical_recommendation}.

Dans la deuxième section, nous avons illustré les liens étroits qui relient la
proportion booléenne et la proportion arithmétique via divers considérations
géométriques. En particulier, nous avons illustré le fait que ces deux
proportions mettent en jeu les quatre sommets d'un parallélogramme soit dans
$\mathbb{B}^m$, soit dans $\mathbb{R}^m$.


Dans la troisième section, nous nous sommes intéressés à la résolution
d'équations analogiques, ainsi qu'au principe d'inférence analogique. En résolvant
une équation analogique, nous sommes capables d'inférer des propriétés
relatives à l'un des quatre éléments de la proportion. Ce processus est en fait
une sorte de généralisation de la fameuse règle de trois: $\frac{4}{2} =
\frac{6}{x} \implies x =3$. Nous avons appliqué cet outil inférentiel à un
problème de classification, en se reposant sur le principe d'inférence
analogique qui stipule que lorsque quatre éléments sont en proportions, alors
cela doit aussi être le cas pour leurs labels respectifs.
