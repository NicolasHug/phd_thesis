\initial{L}e raisonnement par analogie est reconnu comme une des principales
caractéristiques de l'intelligence humaine. En tant que tel, il a pendant
longtemps été étudié par les philosophes et les psychologues, mais de récents
travaux s'intéressent aussi à sa modélisation d'un point de vue formel à l'aide
de proportions analogiques, permettant l'implémentation de programmes
informatiques. Nous nous intéressons ici à l'utilisation des proportions
analogiques à des fins prédictives, dans un contexte d'apprentissage
artificiel.

Dans de récents travaux, les classifieurs analogiques ont montré qu'ils sont
capables d'obtenir d'excellentes performances sur certains problèmes
artificiels où d'autres techniques traditionnelles d'apprentissage se montrent
beaucoup moins efficaces. Partant de cette observation empirique, cette thèse
s'intéresse à deux axes principaux de recherche. Le premier sera d'étudier les
classifieurs analogiques d'un point de vue théorique, car jusqu'à présent
ceux-ci n'étaient connus que grâce à leurs définitions algorithmiques. Les
propriétés théoriques qui découleront nous permettrons de comprendre plus
précisément leurs forces, ainsi que leurs faiblesses. Le second axe de
recherche sera de confronter le raisonnement par proportion analogique à des
applications pratiques, afin d'étudier la viabilité de l'approche analogique
sur des problèmes ``réels''.

Comme domaine d'application, nous avons choisi celui des systèmes de
recommandation. On reproche souvent à ces derniers de manquer de nouveauté ou
de surprise dans les recommandations qui sont adressées aux utilisateurs. Le
raisonnement par analogie, capable de mettre en relation des objets en
apparence différents, nous est apparu comme un outil potentiel pour répondre à
ce problème. Nos expériences montreront que les systèmes analogiques ont
tendance à produire des recommandations d'une qualité comparable à celle des
méthodes existantes, mais que leur complexité algorithmique cubique les
pénalise fortement pour prétendre à des applications pratiques où le temps de
calcul est une des contraintes principales.

Du côté théorique, une contribution majeure de cette thèse est de proposer une
définition fonctionnelle des classifieurs analogiques, qui a la particularité
d'unifier les approches préexistantes. Cette définition fonctionnelle nous
permettra de clairement identifier les liens sous-jacents entre l'approche
analogique et l'approche par $k$ plus proches voisins, tant au niveau
algorithmique de haut niveau qu'au niveau des propriétés théoriques (taux
d'erreur notamment). De plus, nous avons été capables d'identifier un critère
qui rend l'application de notre principe d'inférence analogique parfaitement
certaine (c'est-à-dire sans erreurs),  exhibant ainsi les propriétés linéaires
du raisonnement par analogie.

