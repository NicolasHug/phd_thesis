% Oscar's command (it works): Fills blank pages until next odd-numbered page.
% Used to emulate single-sided frontmatter. This will work for title, abstract
% and declaration. Though the contents sections will each start on an
% odd-numbered page they will spill over onto the even-numbered pages if
% extending beyond one page (hopefully, this is ok).
\newcommand{\clearemptydoublepage}{\newpage{\thispagestyle{empty}\cleardoublepage}}

% My caption style
\newcommand{\mycaption}[2][\@empty]{
	\captionnamefont{\scshape}
	\changecaptionwidth
	\captionwidth{0.9\linewidth}
	\captiondelim{.\:}
	\indentcaption{0.75cm}
	\captionstyle[\centering]{}
	\setlength{\belowcaptionskip}{10pt}
	\ifx \@empty#1 \caption{#2}\else \caption[#1]{#2}
}

% My subcaption style
\newcommand{\mysubcaption}[2][\@empty]{
	\subcaptionsize{\small}
	\hangsubcaption
	\subcaptionlabelfont{\rmfamily}
	\sidecapstyle{\raggedright}
	\setlength{\belowcaptionskip}{10pt}
	\ifx \@empty#1 \subcaption{#2}\else \subcaption[#1]{#2}
}

%An initial of the very first character of the content
\newcommand{\initial}[1]{%
	\lettrine[lines=3,lhang=0.33,nindent=0em]{
		\color{gray}
     		{\textsc{#1}}}{}}

\newcommand{\aggr}[1]{\underset{#1}{\operatorname{aggr}}\;}
\newcommand{\rui}{r_{ui}}
\newcommand{\ruj}{r_{uj}}
\newcommand{\rvi}{r_{vi}}
\newcommand{\predrui}{\hat{r}_{ui}}
\newcommand{\predruj}{\hat{r}_{uj}}
\newcommand{\Iu}{I_u}
\newcommand{\Ui}{U_i}
\newcommand{\Uij}{U_{ij}}
\newcommand{\Iuv}{I_{uv}}
\newcommand{\Rtrain}{R_{\text{train}}}
\newcommand{\Rtest}{R_{\text{test}}}
\newcommand*\albl[1]{\overline{#1}}
\newcommand*\nan[0]{1\mbox{-nan}} %pour 1nan
\newcommand*\knan[0]{k\mbox{-nan}} %pour knan
\newcommand*\NaN[0]{\mbox{NaN}} %pour NaN
\newcommand*\nn[0]{1\mbox{-nn}} %pour 1nn
\newcommand*\NN[0]{\mbox{NN}} %pour NN
\newcommand*\acc[0]{\mbox{Acc}} %pour Acc
\newcommand\given[1][]{\:#1\vert\:} %pour proba conditionnelle
% pour equal by def
\newcommand\eqdef{\mathrel{\overset{\makebox[0pt]{\mbox{\normalfont\tiny\sffamily
def}}}{=}}}
\newcommand\numberthis{\addtocounter{equation}{1}\tag{\theequation}} % sais pas
\newcommand*\aext[2]{\mathbf{E}_{#1,#2}} % Analogical extension
\newcommand*\esf[0]{\mathbf{E}_{S,f}} % Analogical extension
\newcommand*\esfs[0]{\mathbf{E}_{S, f}^*} % Analogical extension star
\newcommand*\aroot[3]{\mathbf{R}_{#1, #2}(#3)} % Analogical root
\newcommand*\rsfx[0]{\mathbf{R}_{S,f}(\mathbf{x})} % Analogical extension
\newcommand*\sol[0]{\text{sol}} % Analogical extension

\DeclareMathOperator*{\argmin}{arg\,min}
\DeclareMathOperator*{\argmax}{arg\,max}
\DeclareMathOperator*{\plim}{\mathit{p}-lim}
\DeclareMathOperator{\ess}{ess}

\newtheorem{definition}{Definition}
\newtheorem{proposition}{Proposition}
\newtheorem{property}{Property}
\newtheorem{problem}{Problem}

% For signed quote env
\def\signed #1{{\leavevmode\unskip\nobreak\hfil\penalty50\hskip2em
\hbox{}\nobreak\hfil(#1)%
\parfillskip=0pt \finalhyphendemerits=0 \endgraf}}
\newsavebox\mybox \newenvironment{aquote}[1]
{\savebox\mybox{#1}\begin{quote}}
{\signed{\usebox\mybox}\end{quote}}

