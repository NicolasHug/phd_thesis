\usepackage[ED=MITT-STICIA, Ets=UT3]{tlsflyleaf}

\usepackage{datetime}
\usepackage{ifpdf} % For pdf metadata
% The import command enables each chapter tex file to use relative paths when
% accessing supplementary files. For example, to include
% chapters/brewing/images/figure1.png from chapters/brewing/brewing.tex we can
% use \includegraphics{images/figure1} instead of
% \includegraphics{chapters/brewing/images/figure1}
\usepackage{import}

\usepackage[]{lipsum}
\usepackage{color} % Creates coloured text and background
\usepackage{footnote}	% For footnotes
\usepackage{microtype}% Makes pdf look better.
\usepackage{lettrine} % For initial letters
\usepackage[version=0.96]{pgf} % For chapter style
\usepackage{calc, soul} % For Chapter style

% Oscar's command (it works): Fills blank pages until next odd-numbered page.
% Used to emulate single-sided frontmatter. This will work for title, abstract
% and declaration. Though the contents sections will each start on an
% odd-numbered page they will spill over onto the even-numbered pages if
% extending beyond one page (hopefully, this is ok).
\newcommand{\clearemptydoublepage}{\newpage{\thispagestyle{empty}\cleardoublepage}}

% My caption style
\newcommand{\mycaption}[2][\@empty]{
	\captionnamefont{\scshape}
	\changecaptionwidth
	\captionwidth{0.9\linewidth}
	\captiondelim{.\:}
	\indentcaption{0.75cm}
	\captionstyle[\centering]{}
	\setlength{\belowcaptionskip}{10pt}
	\ifx \@empty#1 \caption{#2}\else \caption[#1]{#2}
}

% My subcaption style
\newcommand{\mysubcaption}[2][\@empty]{
	\subcaptionsize{\small}
	\hangsubcaption
	\subcaptionlabelfont{\rmfamily}
	\sidecapstyle{\raggedright}
	\setlength{\belowcaptionskip}{10pt}
	\ifx \@empty#1 \subcaption{#2}\else \subcaption[#1]{#2}
}

%An initial of the very first character of the content
\newcommand{\initial}[1]{%
	\lettrine[lines=3,lhang=0.33,nindent=0em]{
		\color{gray}
     		{\textsc{#1}}}{}}

\def\vdotfill#1{\vtop to0pt{\null \dimen0=#1\baselineskip\advance\dimen0 by-.4ex 
   \kern-1.6ex \cleaders\hbox{\lower.4ex\vbox to1ex{}.}\vskip\dimen0 \vss}}

\newcommand{\aggr}[1]{\underset{#1}{\operatorname{aggr}}\;}
\newcommand{\rui}{r_{ui}}
\newcommand{\ruj}{r_{uj}}
\newcommand{\rvi}{r_{vi}}
\newcommand{\predrui}{\hat{r}_{ui}}
\newcommand{\predruj}{\hat{r}_{uj}}
\newcommand{\Iu}{I_u}
\newcommand{\Ui}{U_i}
\newcommand{\Uij}{U_{ij}}
\newcommand{\Iuv}{I_{uv}}
\newcommand{\Rtrain}{R_{\text{train}}}
\newcommand{\Rtest}{R_{\text{test}}}
\newcommand{\ssim}{\text{sim}} % similarity... \sim already exists
\newcommand{\clonedist}{\text{Clone\_dist}} % clone dist
\newcommand{\clonesim}{\text{Clone\_sim}} % clone sim

\newcommand*\albl[1]{\overline{f}(#1)}
\newcommand*\nan[0]{1\text{-nan}} %pour 1nan
\newcommand*\nanemph[0]{1\emph{-nan}} %pour 1nan dans une def ou une prop
\newcommand*\knan[0]{k\text{-nan}} %pour knan
\newcommand*\knn[0]{k\text{-nn}} %pour knn
\newcommand*\NAN[0]{\mbox{NAN}} %pour NaN
\newcommand*\nn[0]{1\mbox{-nn}} %pour 1nn
\newcommand*\NN[0]{\text{NN}} %pour NN
\newcommand*\acc[0]{\mbox{Acc}} %pour Acc
\newcommand\given[1][]{\:#1\vert\:} %pour proba conditionnelle
% pour equal by def
\newcommand\eqdef{\mathrel{\overset{\makebox[0pt]{\mbox{\normalfont\tiny\sffamily
def}}}{=}}}
\newcommand\numberthis{\addtocounter{equation}{1}\tag{\theequation}} % sais pas
\newcommand*\aext[2]{\mathbf{E}_{#1}^{#2}} % Analogical extension
\newcommand*\esf[0]{\mathbf{E}_{S}^{f}} % Analogical extension
\newcommand*\esfs[0]{\mathbf{E}_{S}^{f*}} % Analogical extension star
\newcommand*\aroot[3]{\mathbf{R}_{#1}^{#2}(#3)} % Analogical root
\newcommand*\rsfx[0]{\mathbf{R}_{S}^{f}(\mathbf{x})} % Analogical extension
\newcommand*\sol[0]{\text{sol}} % solution
\newcommand*\solvable[0]{\text{solvable}} % solvable
\newcommand*\AD[0]{\text{AD}} % Analogical Dissimilarity
\newcommand{\norm}[2]{{\left\lVert#2\right\rVert}_{#1}} % Norm definition
\newcommand*\omegasf[0]{\omega_{S}^{f}} % omega
\newcommand*\surpmax[0]{\text{Surp}^{\text{max}}} % surp max
\newcommand*\surpavg[0]{\text{Surp}^{\text{avg}}} % surp avg
\newcommand*\supp[0]{\text{supp}} % support
\newcommand*\knns[0]{k\text{-NN}^*} % knn star (with baseline)
\newcommand*\FCP[0]{\text{FCP}} % FCP
\newcommand*\bin[0]{\text{bin}} % bin
\newcommand*\binemph[0]{\emph{bin}} % binemph

\DeclareMathOperator*{\argmin}{arg\,min}
\DeclareMathOperator*{\argmax}{arg\,max}
\DeclareMathOperator*{\plim}{\mathit{p}-lim}
\newcommand{\pluseq}{\mathrel{+}=} % +=
\DeclareMathOperator{\ess}{ess}

% For vertical and horizontal bars in matrices
\newcommand*{\vertbar}{\rule[-1ex]{0.5pt}{2.5ex}}
\newcommand*{\horzbar}{\rule[.5ex]{2.5ex}{0.5pt}}

\theoremstyle{plain}  % Will be in italic
\newtheorem{proposition}{Proposition}[chapter]
\newtheorem{property}{Property}[chapter]

\theoremstyle{definition}  % Will not be italic
\newtheorem{definition}{Definition}[chapter]
\newtheorem{example}{Example}[chapter]

% For signed quote env
\def\signed #1{{\leavevmode\unskip\nobreak\hfil\penalty50\hskip2em
\hbox{}\nobreak\hfil(#1)%
\parfillskip=0pt \finalhyphendemerits=0 \endgraf}}
\newsavebox\mybox \newenvironment{aquote}[1]
{\savebox\mybox{#1}\begin{quote}}
{\signed{\usebox\mybox}\end{quote}}

% EXAMPLE environment
\newcounter{testexample}
\def\exampletext{Example} % If English
\NewDocumentEnvironment{testexample}{ O{} }
{
\colorlet{colexam}{black} % Global example color
\newtcolorbox[use counter=testexample]{testexamplebox}{%
    % Example Frame Start
    empty,% Empty previously set parameters
    title={\exampletext: #1},% use \thetcbcounter to access the testexample counter text
    % Attaching a box requires an overlay
    attach boxed title to top left,
       % Ensures proper line breaking in longer titles
       minipage boxed title,
    % (boxed title style requires an overlay)
    boxed title style={empty,size=minimal,toprule=0pt,top=4pt,left=3mm,overlay={}},
    coltitle=colexam,fonttitle=\bfseries,
    before=\par\medskip\noindent,parbox=false,boxsep=0pt,left=3mm,right=0mm,top=2pt,breakable,pad at break=0mm,
       before upper=\csname @totalleftmargin\endcsname0pt, % Use instead of parbox=true. This ensures parskip is inherited by box.
    % Handles box when it exists on one page only
    overlay unbroken={\draw[colexam,line width=.5pt] ([xshift=-0pt]title.north west) -- ([xshift=-0pt]frame.south west); },
    % Handles multipage box: first page
    overlay first={\draw[colexam,line width=.5pt] ([xshift=-0pt]title.north west) -- ([xshift=-0pt]frame.south west); },
    % Handles multipage box: middle page
    overlay middle={\draw[colexam,line width=.5pt] ([xshift=-0pt]frame.north west) -- ([xshift=-0pt]frame.south west); },
    % Handles multipage box: last page
    overlay last={\draw[colexam,line width=.5pt] ([xshift=-0pt]frame.north west) -- ([xshift=-0pt]frame.south west); },%
    }
\begin{testexamplebox}}
{\end{testexamplebox}\endlist}


\newcommand{\repeatcaption}[2]{%
  \renewcommand{\thefigure}{\ref{#1}}%
  \captionsetup{list=no}%
  \caption{#2 (repeated from page \pageref{#1})}%
}


\title{Analogical proportions: How to conquer the world.}
\author{Nicolas Hug}
\defencedate{5 juillet 2017}
\lab{Institut de Recherche en Informatique de Toulouse}

\nboss{3}
\makesomeone{boss}{1}{Gilles Richard}{}{}
\makesomeone{boss}{2}{Mathieu Serrurier}{}{}

\nreferee{2}
\makesomeone{referee}{1}{Antoine Cornuéjols}{}{}
\makesomeone{referee}{2}{Jean Lieber}{}{}

\njudge{5}
\makesomeone{judge}{1}{Antoine Cornuéjols}{Professeur}{Rapporteur}
\makesomeone{judge}{2}{Hélène Fargier}{Directeur de recherche}{Membre du jury}
\makesomeone{judge}{3}{Jean Lieber}{Ma\^itre de conférences}{Rapporteur}
\makesomeone{judge}{4}{Henri Prade}{Directeur de recherche}{Membre du jury}
\makesomeone{judge}{5}{Agn\`es Rico}{Ma\^itre de conférences}{Membre du jury}
\makesomeone{judge}{6}{François Yvon}{Professeur}{Membre du jury}

% pdf file meta data
\pdfinfo{
   /Author (Nicolas Hug)
   /Title (TODO TITLE HERE)
   /Keywords (KEYWORDS HERE)
   /CreationDate (D:\pdfdate)
}

% Reduce widows (the last line of a paragraph at the start of a page) and
% orphans (the first line of paragraph at the end of a page)
\widowpenalty=1000
\clubpenalty=1000

% Declare figure/table as a subfloat.
\newsubfloat{figure}
\newsubfloat{table}

% Better page layout for A4 paper, see memoir manual.
\settrimmedsize{297mm}{210mm}{*}
\setlength{\trimtop}{0pt}
\setlength{\trimedge}{\stockwidth}
\addtolength{\trimedge}{-\paperwidth}
\settypeblocksize{634pt}{448.13pt}{*}
\setulmargins{4cm}{*}{*}
\setlrmargins{*}{*}{1.5}
\setmarginnotes{17pt}{51pt}{\onelineskip}
\setheadfoot{\onelineskip}{2\onelineskip}
\setheaderspaces{*}{2\onelineskip}{*}
\checkandfixthelayout
\frenchspacing

% Note: This is automatically set by memoir class. Nevertheless \OnehalfSpacing
% enables double spacing but leaves single spaced for captions for instance.
\OnehalfSpacing

% Sets numbering division level
\setsecnumdepth{subsection}
\maxsecnumdepth{subsubsection}

% The pages should be numbered consecutively at the bottom centre of the
% page.
\makepagestyle{myvf}
\makeoddfoot{myvf}{}{\thepage}{}
\makeevenfoot{myvf}{}{\thepage}{}
\makeheadrule{myvf}{\textwidth}{\normalrulethickness}
\makeevenhead{myvf}{\small\textsc{\leftmark}}{}{}
\makeoddhead{myvf}{}{}{\small\textsc{\rightmark}}
\pagestyle{myvf}

% Chapter style
\makeatletter
\newlength\dlf@normtxtw
\setlength\dlf@normtxtw{\textwidth}
\newsavebox{\feline@chapter}
\newcommand\feline@chapter@marker[1][4cm]{%
	\sbox\feline@chapter{%
		\resizebox{!}{#1}{\fboxsep=1pt%
			\colorbox{gray}{\color{white}\thechapter}%
		}}%
		\rotatebox{90}{%
			\resizebox{%
				\heightof{\usebox{\feline@chapter}}+\depthof{\usebox{\feline@chapter}}}%
			{!}{\scshape\so\@chapapp}}\quad%
		\raisebox{\depthof{\usebox{\feline@chapter}}}{\usebox{\feline@chapter}}%
}
\newcommand\feline@chm[1][4cm]{%
	\sbox\feline@chapter{\feline@chapter@marker[#1]}%
	\makebox[0pt][c]{% aka \rlap
		\makebox[1cm][r]{\usebox\feline@chapter}%
	}}
\makechapterstyle{daleifmodif}{
	\renewcommand\chapnamefont{\normalfont\Large\scshape\raggedleft\so}
	\renewcommand\chaptitlefont{\normalfont\Large\bfseries\scshape}
	\renewcommand\chapternamenum{} \renewcommand\printchaptername{}
	\renewcommand\printchapternum{\null\hfill\feline@chm[2.5cm]\par}
	\renewcommand\afterchapternum{\par\vskip\midchapskip}
	\renewcommand\printchaptertitle[1]{\color{gray}\chaptitlefont\raggedleft ##1\par}
}
\makeatother
\chapterstyle{daleifmodif}


% Put quotes in italic
\AtBeginEnvironment{quote}{\itshape}

% Reduce space between enumerate and itemize
\setlist[enumerate]{topsep=0pt,itemsep=-.5ex,partopsep=1ex,parsep=1ex}
\setlist[itemize]{topsep=0pt,itemsep=-.5ex,partopsep=1ex,parsep=1ex}

% Glossary entries
\newglossaryentry{U_i}
{
  name={\ensuremath{\Ui}},
  description={is the set of all users that have rated item $i$}
}

\newglossaryentry{U_ij}
{
  name={\ensuremath{\Uij}},
  description={is the set of all users that have rated items $i$ and $j$}
}

\newglossaryentry{I_u}
{
  name={\ensuremath{\Iu}},
  description={is the set of all items rated by user $u$}
}

\newglossaryentry{I_uv}
{
  name={\ensuremath{\Iuv}},
  description={is the set of all items rated by users $u$ and $v$}
}

\newglossaryentry{R}
{
  name={\ensuremath{R}},
  description={is the set of all ratings $r_{ui}$ known by the system}
}

\newglossaryentry{r_ui}
{
  name={\ensuremath{\rui}},
  description={is the rating that user $u$ gave to item $i$}
}

\newglossaryentry{hat{r_ui}}
{
  name={\ensuremath{\predrui}},
  description={is the estimation of $\rui$}
}

% After all the glossary entries
\makeglossaries
