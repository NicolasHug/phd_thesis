\usepackage[ED=MITT-STICIA, Ets=UT3]{tlsflyleaf}

\usepackage{datetime}
\usepackage{ifpdf} % For pdf metadata
% The import command enables each chapter tex file to use relative paths when
% accessing supplementary files. For example, to include
% chapters/brewing/images/figure1.png from chapters/brewing/brewing.tex we can
% use \includegraphics{images/figure1} instead of
% \includegraphics{chapters/brewing/images/figure1}
\usepackage{import}

\usepackage[]{lipsum}
\usepackage[dvipsnames]{xcolor}
\usepackage{footnote}	% For footnotes
\usepackage{microtype}% Makes pdf look better.
\usepackage{lettrine} % For initial letters
\usepackage[version=0.96]{pgf} % For chapter style
\usepackage{calc, soul} % For Chapter style
\usepackage{amsmath, amsfonts, amssymb, amsthm} % For da math stuff
\usepackage[toc, nonumberlist]{glossaries} % For list of symbols
\usepackage{todonotes} % For TODOs
\usepackage[super]{nth} % For 1st, 2nd, etc.
\usepackage{algorithm, algorithmic} % For algorithms
\usepackage{wasysym} % For \permil
\usepackage{enumitem} % for enumerate
\usepackage[hidelinks]{hyperref} % For clickable references
\usepackage{gensymb} % for the degree symbol
\usepackage{mathpartir} % for the \inferrule command
\usepackage{braket} % For sizeable brackets using \Set
\usepackage{blkarray} % For matrices with column names
\usepackage{multirow} % For multiple rows in tables
\usepackage{subcaption} % For subfigure
\usepackage[most]{tcolorbox}
\usepackage{xparse}
